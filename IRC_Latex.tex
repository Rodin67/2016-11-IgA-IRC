\documentclass[11pt,a4paper]{article}\usepackage[]{graphicx}\usepackage[]{color}
%% maxwidth is the original width if it is less than linewidth
%% otherwise use linewidth (to make sure the graphics do not exceed the margin)
\makeatletter
\def\maxwidth{ %
  \ifdim\Gin@nat@width>\linewidth
    \linewidth
  \else
    \Gin@nat@width
  \fi
}
\makeatother

\definecolor{fgcolor}{rgb}{0.345, 0.345, 0.345}
\newcommand{\hlnum}[1]{\textcolor[rgb]{0.686,0.059,0.569}{#1}}%
\newcommand{\hlstr}[1]{\textcolor[rgb]{0.192,0.494,0.8}{#1}}%
\newcommand{\hlcom}[1]{\textcolor[rgb]{0.678,0.584,0.686}{\textit{#1}}}%
\newcommand{\hlopt}[1]{\textcolor[rgb]{0,0,0}{#1}}%
\newcommand{\hlstd}[1]{\textcolor[rgb]{0.345,0.345,0.345}{#1}}%
\newcommand{\hlkwa}[1]{\textcolor[rgb]{0.161,0.373,0.58}{\textbf{#1}}}%
\newcommand{\hlkwb}[1]{\textcolor[rgb]{0.69,0.353,0.396}{#1}}%
\newcommand{\hlkwc}[1]{\textcolor[rgb]{0.333,0.667,0.333}{#1}}%
\newcommand{\hlkwd}[1]{\textcolor[rgb]{0.737,0.353,0.396}{\textbf{#1}}}%
\let\hlipl\hlkwb

\usepackage{framed}
\makeatletter
\newenvironment{kframe}{%
 \def\at@end@of@kframe{}%
 \ifinner\ifhmode%
  \def\at@end@of@kframe{\end{minipage}}%
  \begin{minipage}{\columnwidth}%
 \fi\fi%
 \def\FrameCommand##1{\hskip\@totalleftmargin \hskip-\fboxsep
 \colorbox{shadecolor}{##1}\hskip-\fboxsep
     % There is no \\@totalrightmargin, so:
     \hskip-\linewidth \hskip-\@totalleftmargin \hskip\columnwidth}%
 \MakeFramed {\advance\hsize-\width
   \@totalleftmargin\z@ \linewidth\hsize
   \@setminipage}}%
 {\par\unskip\endMakeFramed%
 \at@end@of@kframe}
\makeatother

\definecolor{shadecolor}{rgb}{.97, .97, .97}
\definecolor{messagecolor}{rgb}{0, 0, 0}
\definecolor{warningcolor}{rgb}{1, 0, 1}
\definecolor{errorcolor}{rgb}{1, 0, 0}
\newenvironment{knitrout}{}{} % an empty environment to be redefined in TeX

\usepackage{alltt}

\usepackage[utf8]{inputenc}
%\usepackage[T1]{fontenc}
\usepackage[frenchb]{babel}
\usepackage{amsmath}
\usepackage{amsfonts}
\usepackage{amssymb}
\usepackage{graphicx}
\usepackage{threeparttable}
\usepackage[top=2cm,bottom=2cm,right=2cm,left=2cm]{geometry}

\author{Nathanael Lapidus, Rodolphe JANTZEN}
\title{Etude IRC - bases}
\IfFileExists{upquote.sty}{\usepackage{upquote}}{}
\begin{document}
\maketitle





\section{Incidence maladie de Berger}

On a de 2010 à 2014 1720 patients qui ont eu une première suppléance. L'incidence était de 3.4 cas pour 100 000 habitants.

  \subsection{Incidence spatiale}

Tableau d'incidence selon la région par standardisation directe pour 100 000 habitants de 2010 à 2014 (selon l'effectif français de 2013) :
\begin{knitrout}
\definecolor{shadecolor}{rgb}{0.969, 0.969, 0.969}\color{fgcolor}\begin{kframe}
\begin{verbatim}
                           Ratio.brut Ratio.ajuste IC.inf IC.sup
Reunion                           5.7          5.8    3.9    8.8
Alsace                            5.4          5.5    4.3    6.8
Auvergne                          4.6          4.5    3.3    5.9
Nord.Pas.de.Calais                4.1          4.2    3.5    5.0
Bretagne                          4.0          3.9    3.2    4.7
Midi.Pyrénées                     3.8          3.8    3.0    4.6
Centre                            3.7          3.6    2.9    4.5
Champagne.Ardenne                 3.6          3.6    2.6    5.0
Pays.de.la.Loire                  3.7          3.6    3.0    4.4
Franche.Comté                     3.6          3.5    2.4    5.0
Rhône.Alpes                       3.5          3.5    3.0    4.1
Bourgogne                         3.6          3.4    2.5    4.5
Basse.Normandie                   3.3          3.2    2.3    4.4
Languedoc.Roussillon              3.2          3.2    2.5    4.1
Picardie                          3.0          2.9    2.1    3.9
Haute.Normandie                   2.8          2.8    2.0    3.8
Ile.de.France                     2.8          2.8    2.5    3.2
Aquitaine                         2.7          2.7    2.1    3.4
Lorraine                          2.7          2.7    2.0    3.6
Limousin                          2.6          2.6    1.5    4.4
Poitou.Charentes                  2.5          2.4    1.7    3.4
Provence.Alpes.Côte.d.Azur        2.4          2.3    1.9    2.8
Corse                             1.5          1.5    0.4    4.0
Guadeloupe                        1.0          1.0    0.2    3.4
Martinique                        0.7          0.7    0.1    3.0
\end{verbatim}
\end{kframe}
\end{knitrout}

  \subsection{Incidence temporelle}

\begin{knitrout}
\definecolor{shadecolor}{rgb}{0.969, 0.969, 0.969}\color{fgcolor}\begin{kframe}
\begin{verbatim}

2010 2011 2012 2013 2014 <NA> 
 322  347  358  360  333    0 

2010 2011 2012 2013 2014 
18.7 20.2 20.8 20.9 19.4 
\end{verbatim}
\end{kframe}
\end{knitrout}





















\end{document}
