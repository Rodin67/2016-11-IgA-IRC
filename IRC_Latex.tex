\documentclass[11pt,a4paper]{article}\usepackage[]{graphicx}\usepackage[]{color}
%% maxwidth is the original width if it is less than linewidth
%% otherwise use linewidth (to make sure the graphics do not exceed the margin)
\makeatletter
\def\maxwidth{ %
  \ifdim\Gin@nat@width>\linewidth
    \linewidth
  \else
    \Gin@nat@width
  \fi
}
\makeatother

\definecolor{fgcolor}{rgb}{0.345, 0.345, 0.345}
\newcommand{\hlnum}[1]{\textcolor[rgb]{0.686,0.059,0.569}{#1}}%
\newcommand{\hlstr}[1]{\textcolor[rgb]{0.192,0.494,0.8}{#1}}%
\newcommand{\hlcom}[1]{\textcolor[rgb]{0.678,0.584,0.686}{\textit{#1}}}%
\newcommand{\hlopt}[1]{\textcolor[rgb]{0,0,0}{#1}}%
\newcommand{\hlstd}[1]{\textcolor[rgb]{0.345,0.345,0.345}{#1}}%
\newcommand{\hlkwa}[1]{\textcolor[rgb]{0.161,0.373,0.58}{\textbf{#1}}}%
\newcommand{\hlkwb}[1]{\textcolor[rgb]{0.69,0.353,0.396}{#1}}%
\newcommand{\hlkwc}[1]{\textcolor[rgb]{0.333,0.667,0.333}{#1}}%
\newcommand{\hlkwd}[1]{\textcolor[rgb]{0.737,0.353,0.396}{\textbf{#1}}}%
\let\hlipl\hlkwb

\usepackage{framed}
\makeatletter
\newenvironment{kframe}{%
 \def\at@end@of@kframe{}%
 \ifinner\ifhmode%
  \def\at@end@of@kframe{\end{minipage}}%
  \begin{minipage}{\columnwidth}%
 \fi\fi%
 \def\FrameCommand##1{\hskip\@totalleftmargin \hskip-\fboxsep
 \colorbox{shadecolor}{##1}\hskip-\fboxsep
     % There is no \\@totalrightmargin, so:
     \hskip-\linewidth \hskip-\@totalleftmargin \hskip\columnwidth}%
 \MakeFramed {\advance\hsize-\width
   \@totalleftmargin\z@ \linewidth\hsize
   \@setminipage}}%
 {\par\unskip\endMakeFramed%
 \at@end@of@kframe}
\makeatother

\definecolor{shadecolor}{rgb}{.97, .97, .97}
\definecolor{messagecolor}{rgb}{0, 0, 0}
\definecolor{warningcolor}{rgb}{1, 0, 1}
\definecolor{errorcolor}{rgb}{1, 0, 0}
\newenvironment{knitrout}{}{} % an empty environment to be redefined in TeX

\usepackage{alltt}

\usepackage[utf8]{inputenc}
%\usepackage[T1]{fontenc}
\usepackage[frenchb]{babel}
\usepackage{amsmath}
\usepackage{amsfonts}
\usepackage{amssymb}
\usepackage{graphicx}
\usepackage{threeparttable}
\usepackage[top=2cm,bottom=2cm,right=2cm,left=2cm]{geometry}
\usepackage{float}
\usepackage{multicol}

%\author{}
\title{Étude IRC à IgA 2010-2014}
\IfFileExists{upquote.sty}{\usepackage{upquote}}{}
\begin{document}

\maketitle

\tableofcontents
~\\























Au total, nous avons 18634 patients différents étant passés au stade d'IRCT (\textgreater 16ans, entre 2010 et 2014) dont 1720 (9.2\%) avec maladie de Berger, 11174 (60\%) avec une néphropathie diabétique, 2596 (13.9\%) avec une glomérulonéphrite chronique et 3144 (16.9\%) avec une PKRD.

Nombre de patients par GNC (ont été supprimées les pathologies "GN primitive avec autre diagnostic histologique" et "Maladies systémiques autres") :

\begin{knitrout}
\definecolor{shadecolor}{rgb}{0.969, 0.969, 0.969}\color{fgcolor}\begin{kframe}
\begin{verbatim}
[1] "GN avec HSF : 1184 (6.4%)"
[1] "GN extra-membraneuse : 437 (2.3%)"
[1] "GN extracapillaire ou endo/extracapillaire : 411 (2.2%)"
[1] "GN membrano-proliférative type 1 : 205 (1.1%)"
[1] "GN membrano-proliférative type 2, dépôts denses : 81 (0.4%)"
[1] "Néphropathie lupique : 207 (1.1%)"
[1] "Purpura rhumatoïde : 71 (0.4%)"
\end{verbatim}
\end{kframe}
\end{knitrout}

\section{Incidence IRCT maladie de Berger}



On a 1720 patients qui ont eu une première suppléance. L'incidence était de 3.4 cas pour 100 000 habitants de 2010 à 2014, soit en moyenne \textbf{6.7 nouveau cas d'IRCT/million d'habitants/an}. 

Dans l'article de Simon, on a une incidence de néphropathie à IgA (donc pas forcément au stade d'IRCT) de 26/million d'habitants/an (\textgreater 20 ans) durant la période 1996-2002 en Côte d'Armor (ici pour cette région on a 6.2 cas pour 1M/an d'IRCT). %Sachant que l’incidence de la Maladie de Berger varie de 10 à 40 nouveaux cas/million d’habitants/an et que 25 à 50\% des cas développent une IRCT.

  \subsection{Incidence spatiale d'IRCT}

Tableau d'incidence selon la région par standardisation directe sur l'âge et le sexe pour 1 million d'habitants de 2010 à 2014 (selon l'effectif français de 2013) pour le 1er traitement de suppléance entre 2010 et 2014 :

% latex table generated in R 3.3.2 by xtable 1.8-2 package
% Tue Feb 14 12:35:16 2017
\begin{table}[H]
\centering
\begingroup\small
\begin{tabular}{rrrrrr}
  \hline
 & Ratio.brut & Ratio.ajuste & IC.inf & IC.sup & annuel \\ 
  \hline
Reunion & 56.6 & 57.9 & 39.5 & 88.0 & 11.6 \\ 
  Alsace & 54.3 & 54.6 & 43.4 & 68.1 & 10.9 \\ 
  Auvergne & 46.5 & 44.6 & 33.3 & 58.9 & 8.9 \\ 
  Nord.Pas.de.Calais & 40.5 & 42.3 & 35.3 & 50.5 & 8.5 \\ 
  Bretagne & 39.7 & 39.0 & 31.8 & 47.3 & 7.8 \\ 
  Midi.Pyrénées & 38.1 & 37.7 & 30.4 & 46.3 & 7.5 \\ 
  Centre & 37.3 & 36.3 & 28.6 & 45.4 & 7.3 \\ 
  Pays.de.la.Loire & 36.6 & 36.4 & 29.8 & 44.0 & 7.3 \\ 
  Champagne.Ardenne & 36.2 & 36.2 & 25.7 & 49.5 & 7.2 \\ 
  Franche.Comté & 36.1 & 35.4 & 24.5 & 49.6 & 7.1 \\ 
  Rhône.Alpes & 34.6 & 35.1 & 30.1 & 40.7 & 7.0 \\ 
  Bourgogne & 35.8 & 33.5 & 24.7 & 44.8 & 6.7 \\ 
  Basse.Normandie & 32.6 & 32.3 & 23.0 & 44.4 & 6.5 \\ 
  Languedoc.Roussillon & 32.4 & 32.4 & 25.3 & 41.0 & 6.5 \\ 
  Picardie & 29.6 & 29.2 & 21.3 & 39.3 & 5.8 \\ 
  Ile.de.France & 28.0 & 28.3 & 25.0 & 32.1 & 5.7 \\ 
  Haute.Normandie & 27.9 & 28.0 & 20.1 & 38.2 & 5.6 \\ 
  Aquitaine & 27.2 & 27.2 & 21.4 & 34.3 & 5.4 \\ 
  Lorraine & 27.3 & 27.2 & 20.3 & 35.9 & 5.4 \\ 
  Limousin & 26.1 & 26.3 & 14.8 & 43.8 & 5.3 \\ 
  Guadeloupe & 9.6 & 10.5 & 2.0 & 34.1 & 5.2 \\ 
  Poitou.Charentes & 24.6 & 24.3 & 17.0 & 33.8 & 4.9 \\ 
  Provence.Alpes.Côte.d.Azur & 23.5 & 23.1 & 18.7 & 28.3 & 4.6 \\ 
  Martinique & 6.5 & 7.1 & 0.8 & 29.7 & 3.6 \\ 
  Corse & 15.0 & 15.1 & 4.1 & 40.5 & 3.0 \\ 
   \hline
\end{tabular}
\endgroup
\end{table}



\textcolor{red}{Pour la Guadeloupe et la Martinique, remplissage de la base qu'à partir de 2013}

% latex table generated in R 3.3.2 by xtable 1.8-2 package
% Tue Feb 14 12:35:17 2017
\begin{table}[H]
\centering
\begingroup\small
\begin{tabular}{lrrrrr}
  \hline
Région & 0 & 1 & NA & 0\% & 1\% \\ 
  \hline
Martinique & 0 & 2 & 0 & 0.0 & 100.0 \\ 
  Rhône-Alpes & 22 & 154 & 0 & 12.5 & 87.5 \\ 
  Picardie & 7 & 38 & 0 & 15.6 & 84.4 \\ 
  Bretagne & 18 & 86 & 0 & 17.3 & 82.7 \\ 
  Lorraine & 9 & 43 & 0 & 17.3 & 82.7 \\ 
  Limousin & 3 & 13 & 0 & 18.8 & 81.2 \\ 
  Poitou-Charentes & 7 & 29 & 0 & 19.4 & 80.6 \\ 
  Champagne-Ardenne & 9 & 30 & 0 & 23.1 & 76.9 \\ 
  Haute-Normandie & 11 & 30 & 0 & 26.8 & 73.2 \\ 
  Aquitaine & 20 & 54 & 0 & 27.0 & 73.0 \\ 
  Nord-Pas-de-Calais & 36 & 93 & 0 & 27.9 & 72.1 \\ 
  Basse-Normandie & 11 & 28 & 0 & 28.2 & 71.8 \\ 
  Midi-Pyrénées & 26 & 66 & 0 & 28.3 & 71.7 \\ 
  Bourgogne & 14 & 34 & 0 & 29.2 & 70.8 \\ 
  Auvergne & 17 & 35 & 0 & 32.7 & 67.3 \\ 
  Guadeloupe & 1 & 2 & 0 & 33.3 & 66.7 \\ 
  Alsace & 28 & 54 & 0 & 34.1 & 65.9 \\ 
  Centre & 28 & 49 & 0 & 36.4 & 63.6 \\ 
  Pays de la Loire & 39 & 67 & 0 & 36.8 & 63.2 \\ 
  Ile-de-France & 98 & 166 & 0 & 37.1 & 62.9 \\ 
  Franche-Comté & 13 & 21 & 0 & 38.2 & 61.8 \\ 
  Reunion & 15 & 20 & 0 & 42.9 & 57.1 \\ 
  Corse & 2 & 2 & 0 & 50.0 & 50.0 \\ 
  Languedoc-Roussillon & 36 & 36 & 0 & 50.0 & 50.0 \\ 
  Provence-Alpes-Côte d'Azur & 48 & 47 & 0 & 50.5 & 49.5 \\ 
  ETRANGER & 1 & 0 & 0 & 100.0 & 0.0 \\ 
  Mayotte & 1 & 0 & 0 & 100.0 & 0.0 \\ 
  POLYNESIE FRANCAISE & 1 & 0 & 0 & 100.0 & 0.0 \\ 
  Guyane & 0 & 0 & 0 &  &  \\ 
   \hline
\end{tabular}
\endgroup
\caption{Nombre de PBR par région} 
\end{table}


\begin{knitrout}
\definecolor{shadecolor}{rgb}{0.969, 0.969, 0.969}\color{fgcolor}
\includegraphics[width=\maxwidth]{figure/unnamed-chunk-5-1} 

\end{knitrout}

Tableau d'incidence selon le département par standardisation directe pour 1 million d'habitants de 2010 à 2014 (selon l'effectif français de 2013) :
% latex table generated in R 3.3.2 by xtable 1.8-2 package
% Tue Feb 14 12:35:17 2017
\begin{table}[H]
\centering
\begingroup\small
\begin{tabular}{rrrrrr}
  \hline
 & Ratio.brut & Ratio.ajuste & IC.inf & IC.sup & annuel \\ 
  \hline
Bas.Rhin & 68.9 & 70.3 & 53.9 & 90.9 & 14.1 \\ 
  Réunion & 56.6 & 57.9 & 39.5 & 88.0 & 11.6 \\ 
  Morbihan & 58.6 & 55.6 & 38.7 & 78.4 & 11.1 \\ 
  Savoie & 55.7 & 54.9 & 33.0 & 86.3 & 11.0 \\ 
  Tarn.et.Garonne & 55.0 & 54.7 & 27.1 & 101.6 & 10.9 \\ 
  Hautes.Pyrénées & 57.7 & 53.2 & 26.1 & 101.2 & 10.6 \\ 
  Nord & 49.6 & 52.8 & 42.9 & 64.5 & 10.6 \\ 
  Haute.Marne & 53.8 & 51.8 & 22.3 & 107.6 & 10.4 \\ 
  Haute.Loire & 54.6 & 51.5 & 24.4 & 100.3 & 10.3 \\ 
  Cantal & 56.6 & 50.8 & 20.3 & 115.7 & 10.2 \\ 
  Ardèche & 50.1 & 48.5 & 25.5 & 86.9 & 9.7 \\ 
  Côte.d.Or & 48.7 & 48.3 & 29.9 & 74.2 & 9.7 \\ 
  Puy.de.Dôme & 47.4 & 47.0 & 30.4 & 69.7 & 9.4 \\ 
  Tarn & 48.1 & 46.7 & 25.8 & 79.8 & 9.3 \\ 
  Indre & 47.6 & 46.2 & 20.8 & 94.0 & 9.2 \\ 
  Gers & 50.7 & 45.4 & 19.1 & 100.8 & 9.1 \\ 
  Loire & 47.8 & 45.5 & 30.5 & 65.8 & 9.1 \\ 
  Aude & 47.0 & 44.4 & 24.0 & 77.6 & 8.9 \\ 
  Sarthe & 46.4 & 44.7 & 27.7 & 69.0 & 8.9 \\ 
  Meuse & 45.1 & 44.0 & 17.6 & 93.9 & 8.8 \\ 
  Lozère & 47.3 & 43.0 & 8.6 & 143.7 & 8.6 \\ 
  Jura & 43.0 & 41.7 & 19.0 & 82.7 & 8.3 \\ 
  Oise & 40.9 & 41.0 & 26.7 & 61.2 & 8.2 \\ 
  Charente & 41.2 & 40.5 & 20.8 & 73.2 & 8.1 \\ 
  Vendée & 43.8 & 40.4 & 25.5 & 62.3 & 8.1 \\ 
  Marne & 39.3 & 39.3 & 23.3 & 62.7 & 7.9 \\ 
  Ariège & 39.7 & 39.2 & 12.3 & 102.1 & 7.8 \\ 
  Haute.Vienne & 38.7 & 39.2 & 20.1 & 69.7 & 7.8 \\ 
  Rhône & 37.5 & 39.2 & 29.3 & 51.5 & 7.8 \\ 
  Aveyron & 39.1 & 38.6 & 17.4 & 78.9 & 7.7 \\ 
  Doubs & 37.6 & 37.5 & 21.5 & 61.5 & 7.5 \\ 
  Orne & 38.4 & 36.9 & 16.7 & 73.0 & 7.4 \\ 
  Seine.Saint.Denis & 35.6 & 36.8 & 26.3 & 51.4 & 7.4 \\ 
  Val.de.Marne & 36.5 & 36.8 & 26.1 & 51.0 & 7.4 \\ 
  Var & 38.9 & 37.0 & 25.3 & 52.7 & 7.4 \\ 
  Vosges & 36.1 & 37.0 & 18.3 & 67.9 & 7.4 \\ 
  Hérault & 34.8 & 35.7 & 24.2 & 50.8 & 7.1 \\ 
  Ain & 35.1 & 35.0 & 20.3 & 56.9 & 7.0 \\ 
  Calvados & 34.3 & 34.8 & 20.9 & 54.5 & 7.0 \\ 
  Drôme & 35.5 & 34.8 & 19.0 & 59.4 & 7.0 \\ 
  Paris & 33.6 & 35.2 & 26.9 & 45.5 & 7.0 \\ 
  Allier & 35.2 & 34.6 & 16.3 & 66.9 & 6.9 \\ 
  Finistère & 35.4 & 34.5 & 22.5 & 51.1 & 6.9 \\ 
  Mayenne & 37.2 & 34.4 & 15.6 & 67.9 & 6.9 \\ 
  Haute.Saône & 36.5 & 34.1 & 13.7 & 75.1 & 6.8 \\ 
  Ille.et.Vilaine & 33.5 & 34.0 & 22.4 & 49.9 & 6.8 \\ 
  Loire.Atlantique & 33.3 & 34.0 & 23.7 & 47.5 & 6.8 \\ 
  Pyrénées.Atlantiques & 34.7 & 34.0 & 20.4 & 53.9 & 6.8 \\ 
  Ardennes & 35.6 & 33.4 & 14.4 & 68.3 & 6.7 \\ 
  Lot.et.Garonne & 29.2 & 32.9 & 14.1 & 66.7 & 6.6 \\ 
   \hline
\end{tabular}
\endgroup
\end{table}
% latex table generated in R 3.3.2 by xtable 1.8-2 package
% Tue Feb 14 12:35:17 2017
\begin{table}[H]
\centering
\begingroup\small
\begin{tabular}{rrrrrr}
  \hline
 & Ratio.brut & Ratio.ajuste & IC.inf & IC.sup & annuel \\ 
  \hline
Dordogne & 31.6 & 32.3 & 15.9 & 61.1 & 6.5 \\ 
  Haut.Rhin & 32.8 & 32.3 & 19.7 & 50.9 & 6.5 \\ 
  Seine.Maritime & 30.9 & 31.6 & 21.4 & 44.9 & 6.3 \\ 
  Côtes.d.Armor & 33.1 & 30.9 & 17.6 & 52.3 & 6.2 \\ 
  Landes & 30.8 & 30.3 & 14.4 & 59.1 & 6.1 \\ 
  Haute.Garonne & 28.6 & 29.5 & 19.8 & 42.4 & 5.9 \\ 
  Meurthe.et.Moselle & 28.7 & 29.4 & 17.1 & 47.7 & 5.9 \\ 
  Yonne & 32.7 & 29.7 & 13.5 & 59.8 & 5.9 \\ 
  Deux.Sèvres & 30.0 & 28.8 & 13.1 & 57.4 & 5.8 \\ 
  Hauts.de.Seine & 27.8 & 28.8 & 19.9 & 40.6 & 5.8 \\ 
  Loir.et.Cher & 29.8 & 29.2 & 12.5 & 60.5 & 5.8 \\ 
  Seine.et.Marne & 28.5 & 28.9 & 19.4 & 42.3 & 5.8 \\ 
  Maine.et.Loire & 28.6 & 28.5 & 16.9 & 45.2 & 5.7 \\ 
  Saône.et.Loire & 28.6 & 27.3 & 14.4 & 48.9 & 5.5 \\ 
  Loiret & 26.6 & 26.3 & 14.4 & 44.3 & 5.3 \\ 
  Guadeloupe & 9.6 & 10.5 & 2.0 & 34.1 & 5.2 \\ 
  Manche & 27.1 & 26.0 & 12.9 & 48.4 & 5.2 \\ 
  Vaucluse & 27.2 & 26.1 & 13.4 & 46.3 & 5.2 \\ 
  Isère & 25.7 & 25.7 & 16.6 & 38.3 & 5.1 \\ 
  Gard & 25.4 & 25.0 & 13.9 & 41.9 & 5.0 \\ 
  Aisne & 25.7 & 24.3 & 12.1 & 44.5 & 4.9 \\ 
  Pas.de.Calais & 24.3 & 24.7 & 16.4 & 36.2 & 4.9 \\ 
  Val.d.Oise & 25.1 & 24.5 & 15.5 & 38.1 & 4.9 \\ 
  Hautes.Alpes & 26.5 & 23.9 & 4.9 & 83.6 & 4.8 \\ 
  Pyrénées.Orientales & 23.9 & 23.9 & 10.9 & 46.9 & 4.8 \\ 
  Lot & 20.6 & 22.9 & 4.5 & 78.0 & 4.6 \\ 
  Gironde & 21.3 & 21.6 & 14.1 & 31.7 & 4.3 \\ 
  Nièvre & 27.8 & 21.3 & 6.9 & 58.5 & 4.3 \\ 
  Eure & 21.5 & 20.8 & 10.0 & 39.5 & 4.2 \\ 
  Aube & 20.4 & 20.6 & 6.7 & 48.7 & 4.1 \\ 
  Bouches.du.Rhône & 19.9 & 20.3 & 13.9 & 28.6 & 4.1 \\ 
  Cher & 23.4 & 20.4 & 7.5 & 48.4 & 4.1 \\ 
  Corrèze & 19.9 & 19.4 & 5.1 & 56.4 & 3.9 \\ 
  Eure.et.Loir & 20.5 & 19.4 & 7.8 & 41.5 & 3.9 \\ 
  Moselle & 19.9 & 19.4 & 11.3 & 32.4 & 3.9 \\ 
  Martinique & 6.5 & 7.1 & 0.8 & 29.7 & 3.6 \\ 
  Territoire.de.Belfort & 17.4 & 17.8 & 2.2 & 68.2 & 3.6 \\ 
  Somme & 17.5 & 17.7 & 7.6 & 35.2 & 3.5 \\ 
  Vienne & 17.1 & 17.7 & 6.5 & 38.9 & 3.5 \\ 
  Charente.Maritime & 17.2 & 16.8 & 7.6 & 33.3 & 3.4 \\ 
  Indre.et.Loire & 16.5 & 16.7 & 7.2 & 33.3 & 3.3 \\ 
  Corse.du.Sud & 16.1 & 15.3 & 1.8 & 61.3 & 3.1 \\ 
  Yvelines & 15.4 & 15.4 & 9.0 & 25.3 & 3.1 \\ 
  Essonne & 15.4 & 15.2 & 8.5 & 26.1 & 3.0 \\ 
  Alpes.de.Haute.Provence & 15.1 & 14.4 & 1.7 & 62.8 & 2.9 \\ 
  Alpes.Maritimes & 14.5 & 14.4 & 7.6 & 24.9 & 2.9 \\ 
  Haute.Corse & 14.0 & 14.6 & 1.8 & 56.1 & 2.9 \\ 
  Haute.Savoie & 9.8 & 10.3 & 3.7 & 23.4 & 2.1 \\ 
  Mayotte &  &  &  &  &  \\ 
   \hline
\end{tabular}
\endgroup
\end{table}



\begin{knitrout}
\definecolor{shadecolor}{rgb}{0.969, 0.969, 0.969}\color{fgcolor}
\includegraphics[width=\maxwidth]{figure/unnamed-chunk-7-1} 

\end{knitrout}

\begin{figure}[H]
	\centering
	\includegraphics[width=0.75\linewidth]{C:/Users/Rodolphe/Documents/IRC/carte_densite}
	\caption{Densité néphrologues}
	\label{fig:cartedensite}
\end{figure}

  \subsection{Incidence temporelle d'IRCT (année du 1er traitement de suppléance)}

Pour 1 millions d'habitants français (\textgreater 15 ans) :

% latex table generated in R 3.3.2 by xtable 1.8-2 package
% Tue Feb 14 12:35:18 2017
\begin{table}[H]
\centering
\begin{tabular}{lrrrrr}
  \hline
 & 2010 & 2011 & 2012 & 2013 & 2014 \\ 
  \hline
Berger & 322.0 & 347.0 & 358 & 360 & 333.0 \\ 
  Total France & 50482295.0 & 50701229.0 & 50930784 & 51192770 & 51421008.0 \\ 
  Cas pour 1M d'hab & 6.4 & 6.8 & 7 & 7 & 6.5 \\ 
   \hline
\end{tabular}
\end{table}


\begin{knitrout}
\definecolor{shadecolor}{rgb}{0.969, 0.969, 0.969}\color{fgcolor}\begin{kframe}
\begin{verbatim}

	Pearson's Chi-squared test

data:  inc_iga
X-squared = 2.8618, df = 4, p-value = 0.5812
\end{verbatim}
\end{kframe}
\end{knitrout}

% latex table generated in R 3.3.2 by xtable 1.8-2 package
% Tue Feb 14 12:35:19 2017
\begin{table}[H]
\centering
\begin{tabular}{lrrrrr}
  \hline
Année & 0 & 1 & NA & 0\% & 1\% \\ 
  \hline
2010 & 98 & 224 & 0 & 30.4 & 69.6 \\ 
  2011 & 101 & 246 & 0 & 29.1 & 70.9 \\ 
  2012 & 123 & 235 & 0 & 34.4 & 65.6 \\ 
  2013 & 109 & 251 & 0 & 30.3 & 69.7 \\ 
  2014 & 90 & 243 & 0 & 27.0 & 73.0 \\ 
   \hline
\end{tabular}
\caption{Nombre de PBR par année} 
\end{table}


\begin{knitrout}
\definecolor{shadecolor}{rgb}{0.969, 0.969, 0.969}\color{fgcolor}\begin{kframe}
\begin{verbatim}

	Pearson's Chi-squared test

data:  x[, 2:3]
X-squared = 4.7175, df = 4, p-value = 0.3175
\end{verbatim}
\end{kframe}
\end{knitrout}



Mise en dialyse selon le mois de l'année pour les prises en charge en urgence (n = 336) - Berger:

\begin{knitrout}
\definecolor{shadecolor}{rgb}{0.969, 0.969, 0.969}\color{fgcolor}
\includegraphics[width=\maxwidth]{figure/unnamed-chunk-13-1} 

\end{knitrout}

\begin{knitrout}
\definecolor{shadecolor}{rgb}{0.969, 0.969, 0.969}\color{fgcolor}
\includegraphics[width=\maxwidth]{figure/unnamed-chunk-14-1} 

\end{knitrout}

\begin{figure}[H]
	\centering
	\includegraphics[width=0.65\linewidth]{C:/Users/Rodolphe/Documents/IRC/Documents/4580745_5_04d3_l-episode-grippal-le-plus-important-de-la_dab3b851eb989de590256fa527ffc1b6}
\end{figure}

\section{Etude des caractéristiques cliniques et du devenir des patients}

  \subsection{Caractéristiques cliniques des patients au stade d’IRCT}
  
    \subsubsection{Sexe et âge au stade d’IRCT (1ère suppléance)}

\subsubsection*{Sexe (2010-2014)}

Hommes 1325 (77\%), femmes 395 (23\%)

% latex table generated in R 3.3.2 by xtable 1.8-2 package
% Tue Feb 14 12:35:21 2017
\begin{table}[H]
\centering
\begin{tabular}{rrrrrr}
  \hline
 & 2010 & 2011 & 2012 & 2013 & 2014 \\ 
  \hline
1 & 248 & 260 & 273 & 283 & 261 \\ 
  2 &  74 &  87 &  85 &  77 &  72 \\ 
   \hline
\end{tabular}
\end{table}
% latex table generated in R 3.3.2 by xtable 1.8-2 package
% Tue Feb 14 12:35:21 2017
\begin{table}[H]
\centering
\begin{tabular}{rrrrrr}
  \hline
 & 2010 & 2011 & 2012 & 2013 & 2014 \\ 
  \hline
1 & 77.0 & 74.9 & 76.3 & 78.6 & 78.4 \\ 
  2 & 23.0 & 25.1 & 23.7 & 21.4 & 21.6 \\ 
   \hline
\end{tabular}
\end{table}


\begin{knitrout}
\definecolor{shadecolor}{rgb}{0.969, 0.969, 0.969}\color{fgcolor}\begin{kframe}
\begin{verbatim}

	Pearson's Chi-squared test

data:  table(iga$sex, iga$anirt)
X-squared = 1.8385, df = 4, p-value = 0.7654
\end{verbatim}
\end{kframe}
\end{knitrout}

\subsubsection*{Âge}

% latex table generated in R 3.3.2 by xtable 1.8-2 package
% Tue Feb 14 12:35:21 2017
\begin{table}[ht]
\centering
\begin{tabular}{rrrrrr}
  \hline
minimum & q1 & median & mean & q3 & maximum \\ 
  \hline
16.30 & 40.40 & 53.85 & 53.27 & 65.60 & 93.50 \\ 
   \hline
\end{tabular}
\end{table}
% latex table generated in R 3.3.2 by xtable 1.8-2 package
% Tue Feb 14 12:35:21 2017
\begin{table}[H]
\centering
\begin{tabular}{rrrrrrr}
  \hline
 & Min. & 1st Qu. & Median & Mean & 3rd Qu. & Max. \\ 
  \hline
2010 & 17.20 & 39.72 & 52.05 & 52.28 & 63.27 & 91.10 \\ 
  2011 & 16.30 & 36.90 & 54.20 & 51.74 & 65.25 & 90.00 \\ 
  2012 & 18.40 & 40.90 & 52.85 & 53.52 & 64.68 & 88.20 \\ 
  2013 & 19.40 & 40.90 & 55.45 & 54.59 & 67.03 & 90.30 \\ 
  2014 & 17.90 & 41.40 & 54.30 & 54.13 & 65.90 & 93.50 \\ 
   \hline
\end{tabular}
\end{table}


Distribution des âges :

\begin{knitrout}
\definecolor{shadecolor}{rgb}{0.969, 0.969, 0.969}\color{fgcolor}
\includegraphics[width=\maxwidth]{figure/unnamed-chunk-16-1} 

\end{knitrout}

\begin{knitrout}
\definecolor{shadecolor}{rgb}{0.969, 0.969, 0.969}\color{fgcolor}
\includegraphics[width=\maxwidth]{figure/unnamed-chunk-17-1} 

\includegraphics[width=\maxwidth]{figure/unnamed-chunk-17-2} 

\end{knitrout}

\begin{knitrout}
\definecolor{shadecolor}{rgb}{0.969, 0.969, 0.969}\color{fgcolor}
\includegraphics[width=\maxwidth]{figure/unnamed-chunk-18-1} 

\end{knitrout}

    \subsubsection{Taille, poids et BMI}
  
Taille lors de l'IRCT :
  
% latex table generated in R 3.3.2 by xtable 1.8-2 package
% Tue Feb 14 12:35:22 2017
\begin{table}[ht]
\centering
\begin{tabular}{rrrrrrr}
  \hline
minimum & q1 & median & mean & q3 & maximum & na \\ 
  \hline
131 & 165 & 171 & 170.5 & 176 & 196 & 420 \\ 
   \hline
\end{tabular}
\end{table}


Poids lors de l'IRCT :

% latex table generated in R 3.3.2 by xtable 1.8-2 package
% Tue Feb 14 12:35:22 2017
\begin{table}[ht]
\centering
\begin{tabular}{rrrrrrr}
  \hline
minimum & q1 & median & mean & q3 & maximum & na \\ 
  \hline
11.5 & 63.5 & 72 & 74.01 & 83 & 164 & 294 \\ 
   \hline
\end{tabular}
\end{table}


\begin{knitrout}
\definecolor{shadecolor}{rgb}{0.969, 0.969, 0.969}\color{fgcolor}
\includegraphics[width=\maxwidth]{figure/unnamed-chunk-19-1} 

\end{knitrout}

\begin{knitrout}
\definecolor{shadecolor}{rgb}{0.969, 0.969, 0.969}\color{fgcolor}
\includegraphics[width=\maxwidth]{figure/unnamed-chunk-20-1} 

\end{knitrout}

\begin{knitrout}
\definecolor{shadecolor}{rgb}{0.969, 0.969, 0.969}\color{fgcolor}
\includegraphics[width=\maxwidth]{figure/unnamed-chunk-21-1} 

\end{knitrout}

\begin{knitrout}
\definecolor{shadecolor}{rgb}{0.969, 0.969, 0.969}\color{fgcolor}
\includegraphics[width=\maxwidth]{figure/unnamed-chunk-22-1} 

\end{knitrout}

BMI :

% latex table generated in R 3.3.2 by xtable 1.8-2 package
% Tue Feb 14 12:35:25 2017
\begin{table}[H]
\centering
\begin{tabular}{rrrrrrr}
  \hline
 & 18.5-24.9 & $<$18.5 & 25-29.9 & 30-34.9 & $>$35 & NA \\ 
  \hline
Effectif & 618.0 & 59.0 & 403.0 & 132.0 & 65.0 & 443.0 \\ 
  Pourcentage & 48.4 & 4.6 & 31.6 & 10.3 & 5.1 &  \\ 
   \hline
\end{tabular}
\end{table}


% Age d'arrivée au stade d'IRT selon le BMI :
% 
% <<results="asis">>=
% print(xtable(do.call("rbind", tapply(iga$age, iga$gr_bmi, FUN = summary))), table.placement = "H")
% @

Graphiques relation age - BMI :

\begin{knitrout}
\definecolor{shadecolor}{rgb}{0.969, 0.969, 0.969}\color{fgcolor}
\includegraphics[width=\maxwidth]{figure/unnamed-chunk-23-1} 

\end{knitrout}

\begin{knitrout}
\definecolor{shadecolor}{rgb}{0.969, 0.969, 0.969}\color{fgcolor}
\includegraphics[width=\maxwidth]{figure/unnamed-chunk-24-1} 

\end{knitrout}

\begin{figure}[H]
\begin{knitrout}
\definecolor{shadecolor}{rgb}{0.969, 0.969, 0.969}\color{fgcolor}\begin{kframe}
\begin{verbatim}

Call:
lm(formula = age ~ gr_bmi + DIABn + IDMn + AMIn + ICn + dfg + 
    sex, data = iga)

Residuals:
    Min      1Q  Median      3Q     Max 
-36.502 -11.343  -0.128  11.154  40.940 

Coefficients:
              Estimate Std. Error t value Pr(>|t|)    
(Intercept)    41.4901     2.2734  18.250  < 2e-16 ***
gr_bmi<18.5    -6.1040     2.5577  -2.386 0.017210 *  
gr_bmi25-29.9   3.6332     1.1694   3.107 0.001947 ** 
gr_bmi30-34.9   3.2133     1.7684   1.817 0.069525 .  
gr_bmi>35       0.3713     2.5597   0.145 0.884693    
DIABn           7.3687     1.6840   4.376 1.35e-05 ***
IDMn            6.3016     2.5185   2.502 0.012514 *  
AMIn            9.8694     1.9042   5.183 2.68e-07 ***
ICn             6.2986     1.8099   3.480 0.000525 ***
dfg             0.9873     0.1841   5.364 1.03e-07 ***
sex             0.2892     1.2656   0.228 0.819315    
---
Signif. codes:  0 '***' 0.001 '**' 0.01 '*' 0.05 '.' 0.1 ' ' 1

Residual standard error: 15.62 on 932 degrees of freedom
  (777 observations deleted due to missingness)
Multiple R-squared:  0.1703,	Adjusted R-squared:  0.1614 
F-statistic: 19.12 on 10 and 932 DF,  p-value: < 2.2e-16
\end{verbatim}
\end{kframe}
\end{knitrout}
\end{figure}



\begin{knitrout}
\definecolor{shadecolor}{rgb}{0.969, 0.969, 0.969}\color{fgcolor}
\includegraphics[width=\maxwidth]{figure/unnamed-chunk-27-1} 

\end{knitrout}
~\\

lm(global$\mathdollar$age $\sim$ gr\_bmi + DIABn + IDMn + AMIn + ICn + dfg + sex) :

% latex table generated in R 3.3.2 by xtable 1.8-2 package
% Tue Feb 14 12:35:26 2017
\begin{table}[H]
\centering
\begin{tabular}{rlrrrr}
  \hline
 & .rownames & Estimate & Std..Error & t.value & Pr...t.. \\ 
  \hline
1 & (Intercept) & 47.364 & 0.606 & 78.124 & 0.000 \\ 
  2 & gr\_bmi$<$18.5 & -4.029 & 0.820 & -4.916 & 0.000 \\ 
  3 & gr\_bmi25-29.9 & 2.743 & 0.330 & 8.323 & 0.000 \\ 
  4 & gr\_bmi30-34.9 & 1.870 & 0.396 & 4.727 & 0.000 \\ 
  5 & gr\_bmi$>$35 & -1.587 & 0.464 & -3.420 & 0.001 \\ 
  6 & DIABn & 8.984 & 0.311 & 28.857 & 0.000 \\ 
  7 & IDMn & 1.420 & 0.453 & 3.135 & 0.002 \\ 
  8 & AMIn & 1.899 & 0.337 & 5.628 & 0.000 \\ 
  9 & ICn & 4.513 & 0.339 & 13.318 & 0.000 \\ 
  10 & dfg & 0.679 & 0.045 & 15.126 & 0.000 \\ 
  11 & sex & 1.698 & 0.281 & 6.042 & 0.000 \\ 
   \hline
\end{tabular}
\end{table}


\subsubsection*{Par pathologie (BMI entre 13 et 46)}

\begin{knitrout}
\definecolor{shadecolor}{rgb}{0.969, 0.969, 0.969}\color{fgcolor}
\includegraphics[width=\maxwidth]{figure/unnamed-chunk-29-1} 

\includegraphics[width=\maxwidth]{figure/unnamed-chunk-29-2} 

\includegraphics[width=\maxwidth]{figure/unnamed-chunk-29-3} 

\end{knitrout}

\begin{knitrout}
\definecolor{shadecolor}{rgb}{0.969, 0.969, 0.969}\color{fgcolor}
\includegraphics[width=\maxwidth]{figure/unnamed-chunk-30-1} 

\end{knitrout}



      \subsubsection{Ponction Biopsie Rénale}

% latex table generated in R 3.3.2 by xtable 1.8-2 package
% Tue Feb 14 12:35:29 2017
\begin{table}[H]
\centering
\begin{tabular}{lrrrrr}
  \hline
Pathologie & Non & Oui & Sum & Non\% & Oui\% \\ 
  \hline
GN membrano-proliférative type 1 & 36.0 & 169.0 & 205.0 & 17.6 & 82.4 \\ 
  GN extracapillaire ou endo/extracapillaire & 74.0 & 337.0 & 411.0 & 18.0 & 82.0 \\ 
  GN extra-membraneuse & 99.0 & 338.0 & 437.0 & 22.7 & 77.3 \\ 
  GN membrano-proliférative type 2, dépôts denses & 19.0 & 62.0 & 81.0 & 23.5 & 76.5 \\ 
  GN avec HSF & 328.0 & 856.0 & 1184.0 & 27.7 & 72.3 \\ 
  Néphropathie à dépôts d'IgA & 521.0 & 1199.0 & 1720.0 & 30.3 & 69.7 \\ 
  Purpura rhumatoïde & 23.0 & 48.0 & 71.0 & 32.4 & 67.6 \\ 
  Néphropathie lupique & 69.0 & 138.0 & 207.0 & 33.3 & 66.7 \\ 
  Diabète & 10301.0 & 873.0 & 11174.0 & 92.2 & 7.8 \\ 
  Polykystose rénale de l adulte & 3115.0 & 29.0 & 3144.0 & 99.1 & 0.9 \\ 
   \hline
\end{tabular}
\end{table}


      \subsubsection{Co-morbidités}

\subsubsection*{Diabète}

La prévalence du diabète traité pharmacologiquement en France est estimée à 4,6\% en 2012, tous régimes d’Assurance maladie confondus, 5\% en 2015 (\textit{InVS - SPF}).

% latex table generated in R 3.3.2 by xtable 1.8-2 package
% Tue Feb 14 12:35:29 2017
\begin{table}[H]
\centering
\begin{tabular}{lrrrrrr}
  \hline
Pathologie & Non & Oui & NA & Sum & Non\% & Oui\% \\ 
  \hline
Diabète &  0 & 11174 &  0 & 11174 & 0.0 & 100.0 \\ 
  Purpura rhumatoïde & 47 & 24 &  0 & 71 & 66.2 & 33.8 \\ 
  GN membrano-proliférative type 2, dépôts denses & 63 & 18 &  0 & 81 & 77.8 & 22.2 \\ 
  GN avec HSF & 932 & 244 &  8 & 1184 & 79.3 & 20.7 \\ 
  GN extracapillaire ou endo/extracapillaire & 329 & 78 &  4 & 411 & 80.8 & 19.2 \\ 
  GN extra-membraneuse & 355 & 79 &  3 & 437 & 81.8 & 18.2 \\ 
  GN membrano-proliférative type 1 & 174 & 29 &  2 & 205 & 85.7 & 14.3 \\ 
  Néphropathie à dépôts d'IgA & 1512 & 193 & 15 & 1720 & 88.7 & 11.3 \\ 
  Néphropathie lupique & 184 & 20 &  3 & 207 & 90.2 & 9.8 \\ 
  Polykystose rénale de l adulte & 2863 & 253 & 28 & 3144 & 91.9 & 8.1 \\ 
   \hline
\end{tabular}
\end{table}

% 
% Type de diabète : 
% 
% - 1 : table(iga$TYPDIABn,useNA = "always")[2] (round(prop.table(table(iga$TYPDIABn[!iga$TYPDIABn==0]))*100,1)[1]\%)
% 
% - 2 : table(iga$TYPDIABn,useNA = "always")[3] (round(prop.table(table(iga$TYPDIABn[!iga$TYPDIABn==0]))*100,1)[2]\%)
% 
% <<results="asis">>=
% xtable(addmargins(table(global$liste_longue[!global$TYPDIABn==0], global$TYPDIABn[!global$TYPDIABn==0], useNA = "no")), digits = 0)
% 
% x <- round(prop.table(table(global$liste_longue[!global$TYPDIABn==0], global$TYPDIABn[!global$TYPDIABn==0]),1)*100,1)
% print(xtable(x[order(x[,2], decreasing = T),],digits = 1), table.placement = "H")
% @

Diabète Berger selon le 1er traitement de suppléance :



\subsubsection*{Cirrhose}

Cirrhose et VHB : aucun

Cirrhose et VHC : 4

% latex table generated in R 3.3.2 by xtable 1.8-2 package
% Tue Feb 14 12:35:30 2017
\begin{table}[H]
\centering
\begin{tabular}{lrrrrrr}
  \hline
Pathologie & Non & Oui & NA & Sum & Non\% & Oui\% \\ 
  \hline
GN membrano-proliférative type 2, dépôts denses & 67 &  7 &  7 & 81 & 90.5 & 9.5 \\ 
  Purpura rhumatoïde & 63 &  5 &  3 & 71 & 92.6 & 7.4 \\ 
  Néphropathie à dépôts d'IgA & 1414 & 98 & 208 & 1720 & 93.5 & 6.5 \\ 
  GN membrano-proliférative type 1 & 184 &  9 & 12 & 205 & 95.3 & 4.7 \\ 
  GN extracapillaire ou endo/extracapillaire & 377 & 10 & 24 & 411 & 97.4 & 2.6 \\ 
  Diabète & 10417 & 233 & 524 & 11174 & 97.8 & 2.2 \\ 
  Néphropathie lupique & 183 &  4 & 20 & 207 & 97.9 & 2.1 \\ 
  GN avec HSF & 1087 & 19 & 78 & 1184 & 98.3 & 1.7 \\ 
  GN extra-membraneuse & 399 &  3 & 35 & 437 & 99.3 & 0.7 \\ 
  Polykystose rénale de l adulte & 2706 & 15 & 423 & 3144 & 99.4 & 0.6 \\ 
   \hline
\end{tabular}
\end{table}


\textit{(Remarque : GNMP type 2 : cirrhose influence arrivée en IRCT.)}
~\\

Stade de la cirrhose :

% latex table generated in R 3.3.2 by xtable 1.8-2 package
% Tue Feb 14 12:35:30 2017
\begin{table}[H]
\centering
\begin{tabular}{lrrrrr}
  \hline
Pathologie & Child A & Child B-C & Sum & Non\% & Oui\% \\ 
  \hline
Néphropathie lupique &  0 &  2 &  2 & 0.0 & 100.0 \\ 
  Purpura rhumatoïde &  1 &  3 &  4 & 25.0 & 75.0 \\ 
  GN membrano-proliférative type 1 &  2 &  5 &  7 & 28.6 & 71.4 \\ 
  Néphropathie à dépôts d'IgA & 30 & 49 & 79 & 38.0 & 62.0 \\ 
  GN avec HSF &  6 &  7 & 13 & 46.2 & 53.8 \\ 
  GN extra-membraneuse &  1 &  1 &  2 & 50.0 & 50.0 \\ 
  GN membrano-proliférative type 2, dépôts denses &  3 &  3 &  6 & 50.0 & 50.0 \\ 
  Diabète & 82 & 77 & 159 & 51.6 & 48.4 \\ 
  GN extracapillaire ou endo/extracapillaire &  5 &  3 &  8 & 62.5 & 37.5 \\ 
  Polykystose rénale de l adulte &  8 &  4 & 12 & 66.7 & 33.3 \\ 
   \hline
\end{tabular}
\end{table}


\subsubsection*{Infarctus du myocarde}

% latex table generated in R 3.3.2 by xtable 1.8-2 package
% Tue Feb 14 12:35:30 2017
\begin{table}[H]
\centering
\begin{tabular}{lrrrrrr}
  \hline
Pathologie & Non & Oui & NA & Sum & Non\% & Oui\% \\ 
  \hline
Diabète & 9175 & 1452 & 547 & 11174 & 86.3 & 13.7 \\ 
  Purpura rhumatoïde & 61 &  7 &  3 & 71 & 89.7 & 10.3 \\ 
  GN membrano-proliférative type 2, dépôts denses & 68 &  7 &  6 & 81 & 90.7 & 9.3 \\ 
  GN extra-membraneuse & 374 & 27 & 36 & 437 & 93.3 & 6.7 \\ 
  Néphropathie lupique & 176 & 12 & 19 & 207 & 93.6 & 6.4 \\ 
  Polykystose rénale de l adulte & 2554 & 153 & 437 & 3144 & 94.3 & 5.7 \\ 
  GN avec HSF & 1049 & 61 & 74 & 1184 & 94.5 & 5.5 \\ 
  GN extracapillaire ou endo/extracapillaire & 370 & 20 & 21 & 411 & 94.9 & 5.1 \\ 
  GN membrano-proliférative type 1 & 185 &  9 & 11 & 205 & 95.4 & 4.6 \\ 
  Néphropathie à dépôts d'IgA & 1453 & 60 & 207 & 1720 & 96.0 & 4.0 \\ 
   \hline
\end{tabular}
\end{table}


\subsubsection*{Insuffisance cardiaque}

% latex table generated in R 3.3.2 by xtable 1.8-2 package
% Tue Feb 14 12:35:30 2017
\begin{table}[H]
\centering
\begin{tabular}{lrrrrrr}
  \hline
Pathologie & Non & Oui & NA & Sum & Non\% & Oui\% \\ 
  \hline
Diabète & 7348 & 3301 & 525 & 11174 & 69.0 & 31.0 \\ 
  GN membrano-proliférative type 1 & 156 & 37 & 12 & 205 & 80.8 & 19.2 \\ 
  GN membrano-proliférative type 2, dépôts denses & 62 & 13 &  6 & 81 & 82.7 & 17.3 \\ 
  Purpura rhumatoïde & 59 & 10 &  2 & 71 & 85.5 & 14.5 \\ 
  GN extracapillaire ou endo/extracapillaire & 335 & 54 & 22 & 411 & 86.1 & 13.9 \\ 
  GN extra-membraneuse & 348 & 55 & 34 & 437 & 86.4 & 13.6 \\ 
  GN avec HSF & 978 & 134 & 72 & 1184 & 87.9 & 12.1 \\ 
  Néphropathie lupique & 167 & 22 & 18 & 207 & 88.4 & 11.6 \\ 
  Néphropathie à dépôts d'IgA & 1365 & 149 & 206 & 1720 & 90.2 & 9.8 \\ 
  Polykystose rénale de l adulte & 2457 & 262 & 425 & 3144 & 90.4 & 9.6 \\ 
   \hline
\end{tabular}
\end{table}


Stade de l’IC

% latex table generated in R 3.3.2 by xtable 1.8-2 package
% Tue Feb 14 12:35:30 2017
\begin{table}[H]
\centering
\begin{tabular}{lrrrrr}
  \hline
Pathologie & I-II & III-IV & Sum & I-II\% & III-IV\% \\ 
  \hline
Néphropathie lupique & 12 &  9 & 21 & 57.1 & 42.9 \\ 
  GN membrano-proliférative type 1 & 20 & 13 & 33 & 60.6 & 39.4 \\ 
  GN avec HSF & 78 & 47 & 125 & 62.4 & 37.6 \\ 
  Diabète & 1975 & 980 & 2955 & 66.8 & 33.2 \\ 
  Néphropathie à dépôts d'IgA & 99 & 39 & 138 & 71.7 & 28.3 \\ 
  GN extra-membraneuse & 36 & 13 & 49 & 73.5 & 26.5 \\ 
  GN extracapillaire ou endo/extracapillaire & 38 & 12 & 50 & 76.0 & 24.0 \\ 
  Polykystose rénale de l adulte & 186 & 55 & 241 & 77.2 & 22.8 \\ 
  Purpura rhumatoïde &  7 &  2 &  9 & 77.8 & 22.2 \\ 
  GN membrano-proliférative type 2, dépôts denses &  9 &  2 & 11 & 81.8 & 18.2 \\ 
   \hline
\end{tabular}
\end{table}



\subsubsection*{Artérite des membres inférieurs}

% latex table generated in R 3.3.2 by xtable 1.8-2 package
% Tue Feb 14 12:35:30 2017
\begin{table}[H]
\centering
\begin{tabular}{lrrrrrr}
  \hline
Pathologie & Non & Oui & NA & Sum & Non\% & Oui\% \\ 
  \hline
Diabète & 7032 & 3530 & 612 & 11174 & 66.6 & 33.4 \\ 
  Purpura rhumatoïde & 56 & 10 &  5 & 71 & 84.8 & 15.2 \\ 
  GN extra-membraneuse & 345 & 52 & 40 & 437 & 86.9 & 13.1 \\ 
  GN membrano-proliférative type 1 & 175 & 18 & 12 & 205 & 90.7 & 9.3 \\ 
  GN extracapillaire ou endo/extracapillaire & 351 & 35 & 25 & 411 & 90.9 & 9.1 \\ 
  GN avec HSF & 1008 & 98 & 78 & 1184 & 91.1 & 8.9 \\ 
  Néphropathie à dépôts d'IgA & 1396 & 112 & 212 & 1720 & 92.6 & 7.4 \\ 
  Polykystose rénale de l adulte & 2545 & 160 & 439 & 3144 & 94.1 & 5.9 \\ 
  GN membrano-proliférative type 2, dépôts denses & 71 &  4 &  6 & 81 & 94.7 & 5.3 \\ 
  Néphropathie lupique & 182 &  8 & 17 & 207 & 95.8 & 4.2 \\ 
   \hline
\end{tabular}
\end{table}


Stade de l’AMI

% latex table generated in R 3.3.2 by xtable 1.8-2 package
% Tue Feb 14 12:35:30 2017
\begin{table}[H]
\centering
\begin{tabular}{lrrrrr}
  \hline
Pathologie & I-II & III-IV & Sum & I-II\% & III-IV\% \\ 
  \hline
Néphropathie lupique &  4 &  4 &  8 & 50.0 & 50.0 \\ 
  Purpura rhumatoïde &  5 &  5 & 10 & 50.0 & 50.0 \\ 
  Diabète & 1730 & 1473 & 3203 & 54.0 & 46.0 \\ 
  GN membrano-proliférative type 1 & 10 &  7 & 17 & 58.8 & 41.2 \\ 
  GN avec HSF & 56 & 34 & 90 & 62.2 & 37.8 \\ 
  GN extracapillaire ou endo/extracapillaire & 21 & 11 & 32 & 65.6 & 34.4 \\ 
  GN extra-membraneuse & 31 & 15 & 46 & 67.4 & 32.6 \\ 
  Néphropathie à dépôts d'IgA & 71 & 34 & 105 & 67.6 & 32.4 \\ 
  GN membrano-proliférative type 2, dépôts denses &  3 &  1 &  4 & 75.0 & 25.0 \\ 
  Polykystose rénale de l adulte & 119 & 25 & 144 & 82.6 & 17.4 \\ 
   \hline
\end{tabular}
\end{table}


\subsubsection*{Amputation}

% latex table generated in R 3.3.2 by xtable 1.8-2 package
% Tue Feb 14 12:35:30 2017
\begin{table}[H]
\centering
\begin{tabular}{lrrrrr}
  \hline
Pathologie & Non & Oui & Sum & Non\% & Oui\% \\ 
  \hline
Diabète & 10703 & 471 & 11174 & 95.8 & 4.2 \\ 
  GN membrano-proliférative type 1 & 202 &  3 & 205 & 98.5 & 1.5 \\ 
  Purpura rhumatoïde & 70 &  1 & 71 & 98.6 & 1.4 \\ 
  GN membrano-proliférative type 2, dépôts denses & 80 &  1 & 81 & 98.8 & 1.2 \\ 
  GN extracapillaire ou endo/extracapillaire & 408 &  3 & 411 & 99.3 & 0.7 \\ 
  GN extra-membraneuse & 435 &  2 & 437 & 99.5 & 0.5 \\ 
  Néphropathie à dépôts d'IgA & 1711 &  9 & 1720 & 99.5 & 0.5 \\ 
  Néphropathie lupique & 206 &  1 & 207 & 99.5 & 0.5 \\ 
  GN avec HSF & 1179 &  5 & 1184 & 99.6 & 0.4 \\ 
  Polykystose rénale de l adulte & 3141 &  3 & 3144 & 99.9 & 0.1 \\ 
   \hline
\end{tabular}
\end{table}


\subsubsection*{Accident vasculaire cérébrale (AVC et/ou AIT)}



% latex table generated in R 3.3.2 by xtable 1.8-2 package
% Tue Feb 14 12:35:31 2017
\begin{table}[H]
\centering
\begin{tabular}{lrrrrrr}
  \hline
Pathologie & Non & Oui & NA & Sum & Non\% & Oui\% \\ 
  \hline
Diabète & 9060 & 1447 & 667 & 11174 & 86.2 & 13.8 \\ 
  Purpura rhumatoïde & 63 &  6 &  2 & 71 & 91.3 & 8.7 \\ 
  Néphropathie lupique & 173 & 15 & 19 & 207 & 92.0 & 8.0 \\ 
  GN extracapillaire ou endo/extracapillaire & 356 & 28 & 27 & 411 & 92.7 & 7.3 \\ 
  GN membrano-proliférative type 2, dépôts denses & 69 &  5 &  7 & 81 & 93.2 & 6.8 \\ 
  Polykystose rénale de l adulte & 2474 & 181 & 489 & 3144 & 93.2 & 6.8 \\ 
  GN extra-membraneuse & 368 & 25 & 44 & 437 & 93.6 & 6.4 \\ 
  GN avec HSF & 1023 & 67 & 94 & 1184 & 93.9 & 6.1 \\ 
  GN membrano-proliférative type 1 & 177 & 11 & 17 & 205 & 94.1 & 5.9 \\ 
  Néphropathie à dépôts d'IgA & 1418 & 63 & 239 & 1720 & 95.7 & 4.3 \\ 
   \hline
\end{tabular}
\end{table}


\subsubsection*{VIH}

% latex table generated in R 3.3.2 by xtable 1.8-2 package
% Tue Feb 14 12:35:31 2017
\begin{table}[H]
\centering
\begin{tabular}{lrrrrr}
  \hline
Pathologie & Non & Oui & Sum & Non\% & Oui\% \\ 
  \hline
GN avec HSF & 1131 & 53 & 1184 & 95.5 & 4.5 \\ 
  GN membrano-proliférative type 2, dépôts denses & 79 &  2 & 81 & 97.5 & 2.5 \\ 
  GN membrano-proliférative type 1 & 201 &  4 & 205 & 98.0 & 2.0 \\ 
  Purpura rhumatoïde & 70 &  1 & 71 & 98.6 & 1.4 \\ 
  Néphropathie lupique & 205 &  2 & 207 & 99.0 & 1.0 \\ 
  GN extra-membraneuse & 433 &  4 & 437 & 99.1 & 0.9 \\ 
  Néphropathie à dépôts d'IgA & 1707 & 13 & 1720 & 99.2 & 0.8 \\ 
  Diabète & 11132 & 42 & 11174 & 99.6 & 0.4 \\ 
  GN extracapillaire ou endo/extracapillaire & 410 &  1 & 411 & 99.8 & 0.2 \\ 
  Polykystose rénale de l adulte & 3138 &  6 & 3144 & 99.8 & 0.2 \\ 
   \hline
\end{tabular}
\end{table}


Dont stade SIDA 

% latex table generated in R 3.3.2 by xtable 1.8-2 package
% Tue Feb 14 12:35:31 2017
\begin{table}[H]
\centering
\begin{tabular}{lrrrrr}
  \hline
Pathologie & Non & Oui & Sum & Non\% & Oui\% \\ 
  \hline
GN extracapillaire ou endo/extracapillaire &  0 &  1 &  1 & 0.0 & 100.0 \\ 
  GN membrano-proliférative type 1 &  1 &  3 &  4 & 25.0 & 75.0 \\ 
  GN extra-membraneuse &  2 &  2 &  4 & 50.0 & 50.0 \\ 
  GN avec HSF & 37 & 16 & 53 & 69.8 & 30.2 \\ 
  Diabète & 35 &  7 & 42 & 83.3 & 16.7 \\ 
  Néphropathie à dépôts d'IgA & 12 &  1 & 13 & 92.3 & 7.7 \\ 
  GN membrano-proliférative type 2, dépôts denses &  2 &  0 &  2 & 100.0 & 0.0 \\ 
  Néphropathie lupique &  2 &  0 &  2 & 100.0 & 0.0 \\ 
  Polykystose rénale de l adulte &  6 &  0 &  6 & 100.0 & 0.0 \\ 
  Purpura rhumatoïde &  1 &  0 &  1 & 100.0 & 0.0 \\ 
   \hline
\end{tabular}
\end{table}


\subsubsection*{Ag HBS positif}

% latex table generated in R 3.3.2 by xtable 1.8-2 package
% Tue Feb 14 12:35:31 2017
\begin{table}[H]
\centering
\begin{tabular}{lrrrrr}
  \hline
Pathologie & Non & Oui & Sum & Non\% & Oui\% \\ 
  \hline
GN avec HSF & 1159 & 25 & 1184 & 97.9 & 2.1 \\ 
  GN membrano-proliférative type 1 & 201 &  4 & 205 & 98.0 & 2.0 \\ 
  GN extra-membraneuse & 431 &  6 & 437 & 98.6 & 1.4 \\ 
  GN membrano-proliférative type 2, dépôts denses & 80 &  1 & 81 & 98.8 & 1.2 \\ 
  Néphropathie lupique & 205 &  2 & 207 & 99.0 & 1.0 \\ 
  Diabète & 11088 & 86 & 11174 & 99.2 & 0.8 \\ 
  Néphropathie à dépôts d'IgA & 1708 & 12 & 1720 & 99.3 & 0.7 \\ 
  Polykystose rénale de l adulte & 3124 & 20 & 3144 & 99.4 & 0.6 \\ 
  GN extracapillaire ou endo/extracapillaire & 409 &  2 & 411 & 99.5 & 0.5 \\ 
  Purpura rhumatoïde & 71 &  0 & 71 & 100.0 & 0.0 \\ 
   \hline
\end{tabular}
\end{table}


\subsubsection*{PCR VHC positif}

% latex table generated in R 3.3.2 by xtable 1.8-2 package
% Tue Feb 14 12:35:31 2017
\begin{table}[H]
\centering
\begin{tabular}{lrrrrr}
  \hline
Pathologie & Non & Oui & Sum & Non\% & Oui\% \\ 
  \hline
GN membrano-proliférative type 2, dépôts denses & 75 &  6 & 81 & 92.6 & 7.4 \\ 
  GN membrano-proliférative type 1 & 195 & 10 & 205 & 95.1 & 4.9 \\ 
  GN avec HSF & 1160 & 24 & 1184 & 98.0 & 2.0 \\ 
  Diabète & 10988 & 186 & 11174 & 98.3 & 1.7 \\ 
  GN extracapillaire ou endo/extracapillaire & 405 &  6 & 411 & 98.5 & 1.5 \\ 
  Néphropathie lupique & 204 &  3 & 207 & 98.6 & 1.4 \\ 
  GN extra-membraneuse & 432 &  5 & 437 & 98.9 & 1.1 \\ 
  Néphropathie à dépôts d'IgA & 1709 & 11 & 1720 & 99.4 & 0.6 \\ 
  Polykystose rénale de l adulte & 3125 & 19 & 3144 & 99.4 & 0.6 \\ 
  Purpura rhumatoïde & 71 &  0 & 71 & 100.0 & 0.0 \\ 
   \hline
\end{tabular}
\end{table}


    \subsubsection{Statut Tabagique}

% latex table generated in R 3.3.2 by xtable 1.8-2 package
% Tue Feb 14 12:35:31 2017
\begin{table}[ht]
\centering
\begin{tabular}{lrr}
  \hline
. & Effectif & Pourcentage \\ 
  \hline
NF & 755 & 57.00 \\ 
  Fumeur & 243 & 18.30 \\ 
  EX Fumeur & 327 & 24.70 \\ 
   & 395 &  \\ 
   \hline
\end{tabular}
\end{table}


% latex table generated in R 3.3.2 by xtable 1.8-2 package
% Tue Feb 14 12:35:31 2017
\begin{table}[H]
\centering
\begin{tabular}{rrrrrr}
  \hline
 & NF & Fumeur & EX Fumeur & NA & Sum \\ 
  \hline
Diabète & 5470 & 922 & 2650 & 2132 & 11174 \\ 
  GN avec HSF & 542 & 188 & 254 & 200 & 1184 \\ 
  GN extra-membraneuse & 173 & 49 & 115 & 100 & 437 \\ 
  GN extracapillaire ou endo/extracapillaire & 177 & 50 & 95 & 89 & 411 \\ 
  GN membrano-proliférative type 1 & 99 & 39 & 36 & 31 & 205 \\ 
  GN membrano-proliférative type 2, dépôts denses & 28 & 13 & 26 & 14 & 81 \\ 
  Néphropathie à dépôts d'IgA & 755 & 243 & 327 & 395 & 1720 \\ 
  Néphropathie lupique & 115 & 29 & 17 & 46 & 207 \\ 
  Polykystose rénale de l adulte & 1533 & 291 & 511 & 809 & 3144 \\ 
  Purpura rhumatoïde & 27 & 10 & 19 & 15 & 71 \\ 
  Sum & 8919 & 1834 & 4050 & 3831 & 18634 \\ 
   \hline
\end{tabular}
\end{table}
% latex table generated in R 3.3.2 by xtable 1.8-2 package
% Tue Feb 14 12:35:31 2017
\begin{table}[H]
\centering
\begin{tabular}{rrrr}
  \hline
 & NF & Fumeur & EX Fumeur \\ 
  \hline
GN membrano-proliférative type 2, dépôts denses & 41.8 & 19.4 & 38.8 \\ 
  Purpura rhumatoïde & 48.2 & 17.9 & 33.9 \\ 
  GN extra-membraneuse & 51.3 & 14.5 & 34.1 \\ 
  GN extracapillaire ou endo/extracapillaire & 55.0 & 15.5 & 29.5 \\ 
  GN avec HSF & 55.1 & 19.1 & 25.8 \\ 
  GN membrano-proliférative type 1 & 56.9 & 22.4 & 20.7 \\ 
  Néphropathie à dépôts d'IgA & 57.0 & 18.3 & 24.7 \\ 
  Diabète & 60.5 & 10.2 & 29.3 \\ 
  Polykystose rénale de l adulte & 65.7 & 12.5 & 21.9 \\ 
  Néphropathie lupique & 71.4 & 18.0 & 10.6 \\ 
   \hline
\end{tabular}
\end{table}


\begin{knitrout}
\definecolor{shadecolor}{rgb}{0.969, 0.969, 0.969}\color{fgcolor}\begin{kframe}
\begin{verbatim}

	Pearson's Chi-squared test

data:  x
X-squared = 40.255, df = 18, p-value = 0.001927
\end{verbatim}
\end{kframe}
\end{knitrout}

Goodpasture : sur 28 cas, 12 NF, 3 fumeurs, 7 ex-fumeurs, 6 NA.


    \subsubsection{Créatinine, albumine et hémoglobine}

Pour les patients avec DFG\textgreater15 ou \textgreater20 si IC : suppression DFG et créatinine.

\subsubsection*{Créatininémie initiale ($\mu$mol/l)}

\begin{knitrout}
\definecolor{shadecolor}{rgb}{0.969, 0.969, 0.969}\color{fgcolor}\begin{kframe}
\begin{verbatim}
   Min. 1st Qu.  Median    Mean 3rd Qu.    Max.    NA's 
  288.0   536.5   664.0   737.9   865.2  2793.0     576 
\end{verbatim}
\end{kframe}
\includegraphics[width=\maxwidth]{figure/unnamed-chunk-52-1} 

\end{knitrout}

\subsubsection*{DFG initial}

\begin{knitrout}
\definecolor{shadecolor}{rgb}{0.969, 0.969, 0.969}\color{fgcolor}\begin{kframe}
\begin{verbatim}
   Min. 1st Qu.  Median    Mean 3rd Qu.    Max.    NA's 
  1.600   5.745   7.650   7.910   9.712  19.770     576 
\end{verbatim}
\end{kframe}
\includegraphics[width=\maxwidth]{figure/unnamed-chunk-53-1} 

\end{knitrout}

\begin{knitrout}
\definecolor{shadecolor}{rgb}{0.969, 0.969, 0.969}\color{fgcolor}
\includegraphics[width=\maxwidth]{figure/unnamed-chunk-54-1} 

\end{knitrout}

\begin{knitrout}
\definecolor{shadecolor}{rgb}{0.969, 0.969, 0.969}\color{fgcolor}
\includegraphics[width=\maxwidth]{figure/unnamed-chunk-55-1} 

\end{knitrout}


% latex table generated in R 3.3.2 by xtable 1.8-2 package
% Tue Feb 14 12:35:33 2017
\begin{table}[H]
\centering
\begingroup\small
\begin{tabular}{lrrrrrrr}
  \hline
 & Min. & 1st Qu. & Median & Mean & 3rd Qu. & Max. & NA's \\ 
  \hline
Diabète & 1.6 & 6.8 & 8.9 & 9.1 & 11.3 & 19.9 & 3844 \\ 
  GN membrano-proliférative type 2, dépôts denses & 4.0 & 6.5 & 8.5 & 8.7 & 10.6 & 16.0 & 30 \\ 
  GN membrano-proliférative type 1 & 2.4 & 5.6 & 7.9 & 8.4 & 10.9 & 17.5 & 66 \\ 
  GN extra-membraneuse & 1.5 & 6.1 & 8.0 & 8.4 & 10.3 & 19.6 & 147 \\ 
  Polykystose rénale de l adulte & 1.4 & 6.2 & 7.8 & 8.1 & 9.7 & 20.0 & 1111 \\ 
  Néphropathie lupique & 2.4 & 5.5 & 7.5 & 8.0 & 10.4 & 18.8 & 70 \\ 
  GN avec HSF & 1.5 & 5.8 & 7.7 & 7.9 & 9.9 & 17.2 & 356 \\ 
  Néphropathie à dépôts d'IgA & 1.6 & 5.7 & 7.6 & 7.9 & 9.7 & 19.8 & 576 \\ 
  Purpura rhumatoïde & 2.8 & 5.5 & 7.7 & 7.7 & 9.8 & 13.7 & 24 \\ 
  GN extracapillaire ou endo/extracapillaire & 1.4 & 5.3 & 7.1 & 7.3 & 9.0 & 17.7 & 89 \\ 
   \hline
\end{tabular}
\endgroup
\end{table}


\subsubsection*{Albuminémie initiale}

\begin{knitrout}
\definecolor{shadecolor}{rgb}{0.969, 0.969, 0.969}\color{fgcolor}\begin{kframe}
\begin{verbatim}
   Min. 1st Qu.  Median    Mean 3rd Qu.    Max.    NA's 
   9.00   30.00   34.60   34.05   38.60   58.00     707 
\end{verbatim}
\end{kframe}
\includegraphics[width=\maxwidth]{figure/unnamed-chunk-57-1} 

\end{knitrout}

\subsubsection*{Hémoglobine}

\begin{knitrout}
\definecolor{shadecolor}{rgb}{0.969, 0.969, 0.969}\color{fgcolor}\begin{kframe}
\begin{verbatim}
   Min. 1st Qu.  Median    Mean 3rd Qu.    Max.    NA's 
   4.70    9.10   10.30   10.26   11.40   17.60     400 
\end{verbatim}
\end{kframe}
\includegraphics[width=\maxwidth]{figure/unnamed-chunk-58-1} 

\end{knitrout}

% latex table generated in R 3.3.2 by xtable 1.8-2 package
% Tue Feb 14 12:35:34 2017
\begin{table}[ht]
\centering
\begin{tabular}{lrrrrrrr}
  \hline
 & Min. & 1st Qu. & Median & Mean & 3rd Qu. & Max. & NA's \\ 
  \hline
Diabète & 2.9 & 9.1 & 10.1 & 10.1 & 11.1 & 19.0 & 2079 \\ 
  GN avec HSF & 4.6 & 9.2 & 10.2 & 10.2 & 11.4 & 16.4 & 191 \\ 
  GN extra-membraneuse & 4.3 & 9.3 & 10.3 & 10.2 & 11.3 & 15.3 & 90 \\ 
  GN extracapillaire ou endo/extracapillaire & 3.2 & 8.4 & 9.5 & 9.5 & 10.7 & 15.4 & 74 \\ 
  GN membrano-proliférative type 1 & 6.1 & 8.6 & 9.8 & 9.9 & 11.1 & 14.3 & 37 \\ 
  GN membrano-proliférative type 2, dépôts denses & 6.0 & 8.9 & 9.9 & 10.0 & 11.3 & 14.0 & 19 \\ 
  Néphropathie à dépôts d'IgA & 4.7 & 9.1 & 10.3 & 10.3 & 11.4 & 17.6 & 400 \\ 
  Néphropathie lupique & 5.0 & 8.2 & 9.4 & 9.5 & 10.7 & 13.7 & 48 \\ 
  Polykystose rénale de l adulte & 3.6 & 9.7 & 10.7 & 10.7 & 11.7 & 16.2 & 822 \\ 
  Purpura rhumatoïde & 6.6 & 8.8 & 9.7 & 9.8 & 11.0 & 13.2 & 13 \\ 
   \hline
\end{tabular}
\end{table}



% Nombres d'anémiques :
% 
% <<results="asis">>=
% x <- addmargins(table(iga$sex,iga$gr_HBINI, deparse.level = 2))
% colnames(x) <- c("Anémique","Non anémique",'Somme')
% rownames(x) <- c("Homme","Femme","Somme")
% xtable(x, digits = 0) # Nombre d'anémiques selon le sexe
% xtable(round(prop.table(x[,1:2],1)*100,1), caption = "Pourcentage", digits = 1) 
% @
% 
% <<>>=
% chisq.test(table(iga$sex,iga$gr_HBINI),correct = F) # Chi2 test
% @
% 
% Régression liénaire de l'Hb selon le sexe et le DFG :
% <<results="asis">>=
% iga$gr_HBINI <- as.numeric(as.character(iga$gr_HBINI))
% print(xtable(tidy(summary(lm(iga$HBINI~iga$sex+iga$dfg))), caption = "Régression linéaire simple (Hb selon le sexe et le DFG)", digits = 3), table.placement = "H")
% xtable(confint(lm(iga$HBINI~iga$sex+iga$dfg)), digits = 3)
% @
% 
% <<>>=
% summary(lmer(HBINI ~ dfg + (1|sex), data = iga) )
% @


    \subsubsection{1er traitement de suppléance}

Date = soit date de d'IRCT sinon date de greffe 1. 

Greffe = patients noté comme étant greffé en 1ère suppléance, ceux n'étant que dans la table greffe et ceux ayant eu une greffe dans la 1ère année suivant la 1ère dialyse.
~\\

Pour tous les patients :

% latex table generated in R 3.3.2 by xtable 1.8-2 package
% Tue Feb 14 12:35:34 2017
\begin{table}[H]
\centering
\begin{tabular}{lrrrrr}
  \hline
 & 2010 & 2011 & 2012 & 2013 & 2014 \\ 
  \hline
Dialyse & 3716 & 3649 & 3685 & 3728 & 3636 \\ 
  Greffe & 372 & 376 & 325 & 401 & 375 \\ 
   \hline
\end{tabular}
\end{table}
% latex table generated in R 3.3.2 by xtable 1.8-2 package
% Tue Feb 14 12:35:34 2017
\begin{table}[H]
\centering
\begin{tabular}{lrrrrr}
  \hline
 & 2010 & 2011 & 2012 & 2013 & 2014 \\ 
  \hline
Dialyse & 90.9 & 90.7 & 91.9 & 90.3 & 90.7 \\ 
  Greffe & 9.1 & 9.3 & 8.1 & 9.7 & 9.3 \\ 
   \hline
\end{tabular}
\end{table}


\begin{knitrout}
\definecolor{shadecolor}{rgb}{0.969, 0.969, 0.969}\color{fgcolor}\begin{kframe}
\begin{verbatim}

	Pearson's Chi-squared test

data:  x
X-squared = 23.931, df = 7, p-value = 0.001172
\end{verbatim}
\end{kframe}
\end{knitrout}


Par pathologie :

\begin{multicols}{2}
% latex table generated in R 3.3.2 by xtable 1.8-2 package
% Tue Feb 14 12:35:35 2017
\begin{table}[H]
\centering
\begin{tabular}{rrrrrr}
  \hline
 & 2010 & 2011 & 2012 & 2013 & 2014 \\ 
  \hline
Dialyse & 95.6 & 96.8 & 97.3 & 96.3 & 96.4 \\ 
  Greffe & 4.4 & 3.2 & 2.7 & 3.7 & 3.6 \\ 
   \hline
\end{tabular}
\caption{Pourcentage Diabète} 
\end{table}
% latex table generated in R 3.3.2 by xtable 1.8-2 package
% Tue Feb 14 12:35:35 2017
\begin{table}[H]
\centering
\begin{tabular}{rrrrrr}
  \hline
 & 2010 & 2011 & 2012 & 2013 & 2014 \\ 
  \hline
Dialyse & 85.1 & 85.4 & 90.0 & 85.5 & 90.0 \\ 
  Greffe & 14.9 & 14.6 & 10.0 & 14.5 & 10.0 \\ 
   \hline
\end{tabular}
\caption{Pourcentage GN avec HSF} 
\end{table}
% latex table generated in R 3.3.2 by xtable 1.8-2 package
% Tue Feb 14 12:35:35 2017
\begin{table}[H]
\centering
\begin{tabular}{rrrrrr}
  \hline
 & 2010 & 2011 & 2012 & 2013 & 2014 \\ 
  \hline
Dialyse & 88.0 & 87.1 & 88.5 & 83.1 & 90.2 \\ 
  Greffe & 12.0 & 12.9 & 11.5 & 16.9 & 9.8 \\ 
   \hline
\end{tabular}
\caption{Pourcentage GN extra-membraneuse} 
\end{table}
% latex table generated in R 3.3.2 by xtable 1.8-2 package
% Tue Feb 14 12:35:35 2017
\begin{table}[H]
\centering
\begin{tabular}{rrrrrr}
  \hline
 & 2010 & 2011 & 2012 & 2013 & 2014 \\ 
  \hline
Dialyse & 90.7 & 97.5 & 94.9 & 100.0 & 90.9 \\ 
  Greffe & 9.3 & 2.5 & 5.1 & 0.0 & 9.1 \\ 
   \hline
\end{tabular}
\caption{Pourcentage GN extracapillaire ou endo/extracapillaire} 
\end{table}
% latex table generated in R 3.3.2 by xtable 1.8-2 package
% Tue Feb 14 12:35:35 2017
\begin{table}[H]
\centering
\begin{tabular}{rrrrrr}
  \hline
 & 2010 & 2011 & 2012 & 2013 & 2014 \\ 
  \hline
Dialyse & 89.5 & 82.9 & 89.5 & 91.1 & 93.8 \\ 
  Greffe & 10.5 & 17.1 & 10.5 & 8.9 & 6.2 \\ 
   \hline
\end{tabular}
\caption{Pourcentage GN membrano-proliférative type 1} 
\end{table}
% latex table generated in R 3.3.2 by xtable 1.8-2 package
% Tue Feb 14 12:35:35 2017
\begin{table}[H]
\centering
\begin{tabular}{rrrrrr}
  \hline
 & 2010 & 2011 & 2012 & 2013 & 2014 \\ 
  \hline
Dialyse & 75.0 & 84.6 & 77.8 & 72.2 & 85.7 \\ 
  Greffe & 25.0 & 15.4 & 22.2 & 27.8 & 14.3 \\ 
   \hline
\end{tabular}
\caption{Pourcentage GN membrano-proliférative type 2, dépôts denses} 
\end{table}
% latex table generated in R 3.3.2 by xtable 1.8-2 package
% Tue Feb 14 12:35:35 2017
\begin{table}[H]
\centering
\begin{tabular}{rrrrrr}
  \hline
 & 2010 & 2011 & 2012 & 2013 & 2014 \\ 
  \hline
Dialyse & 76.1 & 71.2 & 76.3 & 78.1 & 72.1 \\ 
  Greffe & 23.9 & 28.8 & 23.7 & 21.9 & 27.9 \\ 
   \hline
\end{tabular}
\caption{Pourcentage Néphropathie à dépôts d'IgA} 
\end{table}
% latex table generated in R 3.3.2 by xtable 1.8-2 package
% Tue Feb 14 12:35:35 2017
\begin{table}[H]
\centering
\begin{tabular}{rrrrrr}
  \hline
 & 2010 & 2011 & 2012 & 2013 & 2014 \\ 
  \hline
Dialyse & 82.5 & 86.7 & 93.2 & 85.7 & 88.9 \\ 
  Greffe & 17.5 & 13.3 & 6.8 & 14.3 & 11.1 \\ 
   \hline
\end{tabular}
\caption{Pourcentage Néphropathie lupique} 
\end{table}
% latex table generated in R 3.3.2 by xtable 1.8-2 package
% Tue Feb 14 12:35:35 2017
\begin{table}[H]
\centering
\begin{tabular}{rrrrrr}
  \hline
 & 2010 & 2011 & 2012 & 2013 & 2014 \\ 
  \hline
Dialyse & 77.4 & 76.1 & 78.3 & 75.0 & 78.2 \\ 
  Greffe & 22.6 & 23.9 & 21.7 & 25.0 & 21.8 \\ 
   \hline
\end{tabular}
\caption{Pourcentage Polykystose rénale de l adulte} 
\end{table}
% latex table generated in R 3.3.2 by xtable 1.8-2 package
% Tue Feb 14 12:35:35 2017
\begin{table}[H]
\centering
\begin{tabular}{rrrrrr}
  \hline
 & 2010 & 2011 & 2012 & 2013 & 2014 \\ 
  \hline
Dialyse & 86.7 & 88.2 & 89.5 & 92.3 & 85.7 \\ 
  Greffe & 13.3 & 11.8 & 10.5 & 7.7 & 14.3 \\ 
   \hline
\end{tabular}
\caption{Pourcentage Purpura rhumatoïde} 
\end{table}

\end{multicols}

\textcolor{red}{Nombre de patients greffés direct n'ayant aucune info sur la greffe toute pathologies confondus : 154 (8.3\%)}
~\\

%\textcolor{red}{XXXXXX} Nombre de greffé sans dialyse et nombre de dialysé après dialyse :

%<<results="asis">>=
%x <- matrix(nrow = 2, ncol = 5, dimnames = list(c("Greffe sans dialyse","Greffe après dialyse"),c("2010","2011","2012","2013","2014")))
%x[1,] <- c(29,34,33,31,40)
%x[2,] <- c(187-29,192-34,155-33,108-31,64-40)
%print(xtable(addmargins(x), digits = 0), table.placement = "H")
%print(xtable(prop.table(x,2)*100, digits = 1), table.placement = "H")
%@

%\textcolor{red}{Attention : le recul n'est pas le même pour ceux de 2010 et 2014, ce qui explique la différence.}
%~\\

%Nombre de dialyse (hémo + péritonéale) comparé aux dialysés totaux en France : (à refaire XXXXX)



    \subsubsection{Contexte de démarrage de dialyse}

  \subsubsection*{Urgence}


    
Premier traitement en réanimation ou urgence IgA :  336 (23.2\%) NA = 273

Premier traitement en réanimation ou urgence diabète :  3310 (32.7\%) NA = 1050

Premier traitement en réanimation ou urgence PKRD :  352 (13.7\%) NA = 578

Premier traitement en réanimation ou urgence GNC :  743 (31.9\%) NA = 269

\begin{knitrout}
\definecolor{shadecolor}{rgb}{0.969, 0.969, 0.969}\color{fgcolor}\begin{kframe}
\begin{verbatim}
[1] "274 ( 23.1 %) : GN avec HSF"
[1] "103 ( 23.6 %) : GN extra-membraneuse"
[1] "184 ( 44.8 %) : GN extracapillaire ou endo/extracapillaire"
[1] "66 ( 32.2 %) : GN membrano-proliférative type 1"
[1] "24 ( 29.6 %) : GN membrano-proliférative type 2, dépôts denses"
[1] "63 ( 30.4 %) : Néphropathie lupique"
[1] "29 ( 40.8 %) : Purpura rhumatoïde"
\end{verbatim}
\end{kframe}
\end{knitrout}

Entrée en urgence selon l'âge :

% latex table generated in R 3.3.2 by xtable 1.8-2 package
% Tue Feb 14 12:35:36 2017
\begin{table}[H]
\centering
\begin{tabular}{rrr}
  \hline
 & 0 & 1 \\ 
  \hline
(0,40] & 69.20 & 30.80 \\ 
  (40,99] & 79.10 & 20.90 \\ 
   \hline
\end{tabular}
\end{table}


\begin{knitrout}
\definecolor{shadecolor}{rgb}{0.969, 0.969, 0.969}\color{fgcolor}\begin{kframe}
\begin{verbatim}

	Pearson's Chi-squared test

data:  table(x$gr_age_2, x$urg_rea)
X-squared = 14.345, df = 1, p-value = 0.0001522
\end{verbatim}
\end{kframe}
\end{knitrout}

La différence entre les 2 groupes d'âge n'est plus significative à 50 ans.

  \subsubsection*{Voie d’abord}

% latex table generated in R 3.3.2 by xtable 1.8-2 package
% Tue Feb 14 12:35:36 2017
\begin{table}[H]
\centering
\begin{tabular}{rrrrr}
  \hline
 & FAV native & Cathéter tunnélisé & Pontage & Autre \\ 
  \hline
PKRD & 74.80 & 19.90 & 1.60 & 3.70 \\ 
  IgA & 59.80 & 33.90 & 0.60 & 5.80 \\ 
  Diabète & 49.10 & 42.80 & 1.80 & 6.30 \\ 
  GNC & 47.30 & 44.00 & 1.10 & 7.60 \\ 
   \hline
\end{tabular}
\caption{Pourcentage} 
\end{table}


% latex table generated in R 3.3.2 by xtable 1.8-2 package
% Tue Feb 14 12:35:36 2017
\begin{table}[H]
\centering
\begin{tabular}{lrrrr}
  \hline
.rownames & FAV.native & Cathéter.tunnélisé & Pontage & Autre \\ 
  \hline
GN avec HSF & 56.80 & 36.50 & 1.30 & 5.40 \\ 
  GN extra-membraneuse & 49.80 & 44.40 & 1.50 & 4.30 \\ 
  GN extracapillaire ou endo/extracapillaire & 32.00 & 56.40 & 0.80 & 10.80 \\ 
  GN membrano-proliférative type 1 & 39.70 & 48.90 & 0.60 & 10.90 \\ 
  GN membrano-proliférative type 2, dépôts denses & 42.60 & 51.50 & 1.50 & 4.40 \\ 
  Néphropathie lupique & 37.30 & 46.80 & 0.60 & 15.20 \\ 
  Purpura rhumatoïde & 33.30 & 52.40 & 0.00 & 14.30 \\ 
   \hline
\end{tabular}
\end{table}


Selon l'âge :

\begin{multicols}{2}
% latex table generated in R 3.3.2 by xtable 1.8-2 package
% Tue Feb 14 12:35:36 2017
\begin{table}[H]
\centering
\begin{tabular}{rrr}
  \hline
 & (0,40] & (40,99] \\ 
  \hline
FAV native & 45.20 & 49.20 \\ 
  Cathéter tunnélisé & 48.50 & 42.70 \\ 
  Pontage & 0.80 & 1.80 \\ 
  Autre & 5.40 & 6.30 \\ 
   \hline
\end{tabular}
\caption{Diabète} 
\end{table}
% latex table generated in R 3.3.2 by xtable 1.8-2 package
% Tue Feb 14 12:35:36 2017
\begin{table}[H]
\centering
\begin{tabular}{rrr}
  \hline
 & (0,40] & (40,99] \\ 
  \hline
FAV native & 53.60 & 57.50 \\ 
  Cathéter tunnélisé & 39.20 & 35.90 \\ 
  Pontage & 0.00 & 1.60 \\ 
  Autre & 7.20 & 5.10 \\ 
   \hline
\end{tabular}
\caption{GN avec HSF} 
\end{table}
% latex table generated in R 3.3.2 by xtable 1.8-2 package
% Tue Feb 14 12:35:36 2017
\begin{table}[H]
\centering
\begin{tabular}{rrr}
  \hline
 & (0,40] & (40,99] \\ 
  \hline
FAV native & 38.50 & 50.80 \\ 
  Cathéter tunnélisé & 46.20 & 44.20 \\ 
  Pontage & 0.00 & 1.70 \\ 
  Autre & 15.40 & 3.30 \\ 
   \hline
\end{tabular}
\caption{GN extra-membraneuse} 
\end{table}
% latex table generated in R 3.3.2 by xtable 1.8-2 package
% Tue Feb 14 12:35:36 2017
\begin{table}[H]
\centering
\begin{tabular}{rrr}
  \hline
 & (0,40] & (40,99] \\ 
  \hline
FAV native & 21.40 & 32.80 \\ 
  Cathéter tunnélisé & 67.90 & 55.40 \\ 
  Pontage & 0.00 & 0.90 \\ 
  Autre & 10.70 & 10.90 \\ 
   \hline
\end{tabular}
\caption{GN extracapillaire ou endo/extracapillaire} 
\end{table}
% latex table generated in R 3.3.2 by xtable 1.8-2 package
% Tue Feb 14 12:35:36 2017
\begin{table}[H]
\centering
\begin{tabular}{rrr}
  \hline
 & (0,40] & (40,99] \\ 
  \hline
FAV native & 58.60 & 35.90 \\ 
  Cathéter tunnélisé & 31.00 & 52.40 \\ 
  Pontage & 0.00 & 0.70 \\ 
  Autre & 10.30 & 11.00 \\ 
   \hline
\end{tabular}
\caption{GN membrano-proliférative type 1} 
\end{table}
% latex table generated in R 3.3.2 by xtable 1.8-2 package
% Tue Feb 14 12:35:36 2017
\begin{table}[H]
\centering
\begin{tabular}{rrr}
  \hline
 & (0,40] & (40,99] \\ 
  \hline
FAV native & 28.60 & 44.30 \\ 
  Cathéter tunnélisé & 57.10 & 50.80 \\ 
  Pontage & 0.00 & 1.60 \\ 
  Autre & 14.30 & 3.30 \\ 
   \hline
\end{tabular}
\caption{GN membrano-proliférative type 2, dépôts denses} 
\end{table}
% latex table generated in R 3.3.2 by xtable 1.8-2 package
% Tue Feb 14 12:35:36 2017
\begin{table}[H]
\centering
\begin{tabular}{rrr}
  \hline
 & (0,40] & (40,99] \\ 
  \hline
FAV native & 49.80 & 62.50 \\ 
  Cathéter tunnélisé & 44.00 & 31.10 \\ 
  Pontage & 0.00 & 0.70 \\ 
  Autre & 6.20 & 5.70 \\ 
   \hline
\end{tabular}
\caption{Néphropathie à dépôts d'IgA} 
\end{table}
% latex table generated in R 3.3.2 by xtable 1.8-2 package
% Tue Feb 14 12:35:36 2017
\begin{table}[H]
\centering
\begin{tabular}{rrr}
  \hline
 & (0,40] & (40,99] \\ 
  \hline
FAV native & 34.40 & 39.20 \\ 
  Cathéter tunnélisé & 44.30 & 48.50 \\ 
  Pontage & 1.60 & 0.00 \\ 
  Autre & 19.70 & 12.40 \\ 
   \hline
\end{tabular}
\caption{Néphropathie lupique} 
\end{table}
% latex table generated in R 3.3.2 by xtable 1.8-2 package
% Tue Feb 14 12:35:36 2017
\begin{table}[H]
\centering
\begin{tabular}{rrr}
  \hline
 & (0,40] & (40,99] \\ 
  \hline
FAV native & 67.30 & 75.20 \\ 
  Cathéter tunnélisé & 23.10 & 19.80 \\ 
  Pontage & 1.00 & 1.60 \\ 
  Autre & 8.70 & 3.50 \\ 
   \hline
\end{tabular}
\caption{Polykystose rénale de l adulte} 
\end{table}
% latex table generated in R 3.3.2 by xtable 1.8-2 package
% Tue Feb 14 12:35:36 2017
\begin{table}[H]
\centering
\begin{tabular}{rrr}
  \hline
 & (0,40] & (40,99] \\ 
  \hline
FAV native & 33.30 & 33.30 \\ 
  Cathéter tunnélisé & 33.30 & 55.60 \\ 
  Pontage & 0.00 & 0.00 \\ 
  Autre & 33.30 & 11.10 \\ 
   \hline
\end{tabular}
\caption{Purpura rhumatoïde} 
\end{table}

\end{multicols}

Selon l'urgence :
\begin{multicols}{2}
% latex table generated in R 3.3.2 by xtable 1.8-2 package
% Tue Feb 14 12:35:37 2017
\begin{table}[H]
\centering
\begin{tabular}{rrr}
  \hline
 & 0 & 1 \\ 
  \hline
FAV native & 61.10 & 25.80 \\ 
  Cathéter tunnélisé & 33.00 & 62.00 \\ 
  Pontage & 1.90 & 1.40 \\ 
  Autre & 4.10 & 10.80 \\ 
   \hline
\end{tabular}
\caption{Diabète} 
\end{table}
% latex table generated in R 3.3.2 by xtable 1.8-2 package
% Tue Feb 14 12:35:37 2017
\begin{table}[H]
\centering
\begin{tabular}{rrr}
  \hline
 & 0 & 1 \\ 
  \hline
FAV native & 71.00 & 22.50 \\ 
  Cathéter tunnélisé & 24.80 & 64.50 \\ 
  Pontage & 1.60 & 0.40 \\ 
  Autre & 2.70 & 12.60 \\ 
   \hline
\end{tabular}
\caption{GN avec HSF} 
\end{table}
% latex table generated in R 3.3.2 by xtable 1.8-2 package
% Tue Feb 14 12:35:37 2017
\begin{table}[H]
\centering
\begin{tabular}{rrr}
  \hline
 & 0 & 1 \\ 
  \hline
FAV native & 60.40 & 26.40 \\ 
  Cathéter tunnélisé & 35.60 & 63.70 \\ 
  Pontage & 2.20 & 0.00 \\ 
  Autre & 1.80 & 9.90 \\ 
   \hline
\end{tabular}
\caption{GN extra-membraneuse} 
\end{table}
% latex table generated in R 3.3.2 by xtable 1.8-2 package
% Tue Feb 14 12:35:37 2017
\begin{table}[H]
\centering
\begin{tabular}{rrr}
  \hline
 & 0 & 1 \\ 
  \hline
FAV native & 50.60 & 13.10 \\ 
  Cathéter tunnélisé & 38.90 & 73.10 \\ 
  Pontage & 0.60 & 1.10 \\ 
  Autre & 10.00 & 12.60 \\ 
   \hline
\end{tabular}
\caption{GN extracapillaire ou endo/extracapillaire} 
\end{table}
% latex table generated in R 3.3.2 by xtable 1.8-2 package
% Tue Feb 14 12:35:37 2017
\begin{table}[H]
\centering
\begin{tabular}{rrr}
  \hline
 & 0 & 1 \\ 
  \hline
FAV native & 52.50 & 19.00 \\ 
  Cathéter tunnélisé & 40.60 & 61.90 \\ 
  Pontage & 1.00 & 0.00 \\ 
  Autre & 5.90 & 19.00 \\ 
   \hline
\end{tabular}
\caption{GN membrano-proliférative type 1} 
\end{table}
% latex table generated in R 3.3.2 by xtable 1.8-2 package
% Tue Feb 14 12:35:37 2017
\begin{table}[H]
\centering
\begin{tabular}{rrr}
  \hline
 & 0 & 1 \\ 
  \hline
FAV native & 53.30 & 18.20 \\ 
  Cathéter tunnélisé & 42.20 & 72.70 \\ 
  Pontage & 2.20 & 0.00 \\ 
  Autre & 2.20 & 9.10 \\ 
   \hline
\end{tabular}
\caption{GN membrano-proliférative type 2, dépôts denses} 
\end{table}
% latex table generated in R 3.3.2 by xtable 1.8-2 package
% Tue Feb 14 12:35:37 2017
\begin{table}[H]
\centering
\begin{tabular}{rrr}
  \hline
 & 0 & 1 \\ 
  \hline
FAV native & 74.60 & 18.40 \\ 
  Cathéter tunnélisé & 20.80 & 69.70 \\ 
  Pontage & 0.60 & 0.70 \\ 
  Autre & 4.00 & 11.20 \\ 
   \hline
\end{tabular}
\caption{Néphropathie à dépôts d'IgA} 
\end{table}
% latex table generated in R 3.3.2 by xtable 1.8-2 package
% Tue Feb 14 12:35:37 2017
\begin{table}[H]
\centering
\begin{tabular}{rrr}
  \hline
 & 0 & 1 \\ 
  \hline
FAV native & 51.20 & 15.30 \\ 
  Cathéter tunnélisé & 40.70 & 55.90 \\ 
  Pontage & 0.00 & 1.70 \\ 
  Autre & 8.10 & 27.10 \\ 
   \hline
\end{tabular}
\caption{Néphropathie lupique} 
\end{table}
% latex table generated in R 3.3.2 by xtable 1.8-2 package
% Tue Feb 14 12:35:37 2017
\begin{table}[H]
\centering
\begin{tabular}{rrr}
  \hline
 & 0 & 1 \\ 
  \hline
FAV native & 82.10 & 35.50 \\ 
  Cathéter tunnélisé & 14.00 & 51.20 \\ 
  Pontage & 1.80 & 0.90 \\ 
  Autre & 2.10 & 12.30 \\ 
   \hline
\end{tabular}
\caption{Polykystose rénale de l adulte} 
\end{table}
% latex table generated in R 3.3.2 by xtable 1.8-2 package
% Tue Feb 14 12:35:37 2017
\begin{table}[H]
\centering
\begin{tabular}{rrr}
  \hline
 & 0 & 1 \\ 
  \hline
FAV native & 48.30 & 11.10 \\ 
  Cathéter tunnélisé & 41.40 & 70.40 \\ 
  Pontage & 0.00 & 0.00 \\ 
  Autre & 10.30 & 18.50 \\ 
   \hline
\end{tabular}
\caption{Purpura rhumatoïde} 
\end{table}

\end{multicols}

  \subsection{Analyse de la survie patient avec comme outcome le décès}
  
  On a la date de décès pour 238 (13.8\%) patients pour la maladie de Berger.
  
  On a la date de décès pour 4586 (41\%) patients pour le diabète.
  
  On a la date de décès pour 371 (11.8\%) patients pour la PKRD.
  
  On a la date de décès pour 592 (22.8\%) patients pour la GNC.

\begin{knitrout}
\definecolor{shadecolor}{rgb}{0.969, 0.969, 0.969}\color{fgcolor}\begin{kframe}
\begin{verbatim}
[1] "213 ( 18 %) : GN avec HSF"
[1] "120 ( 27.5 %) : GN extra-membraneuse"
[1] "130 ( 31.6 %) : GN extracapillaire ou endo/extracapillaire"
[1] "61 ( 29.8 %) : GN membrano-proliférative type 1"
[1] "14 ( 17.3 %) : GN membrano-proliférative type 2, dépôts denses"
[1] "34 ( 16.4 %) : Néphropathie lupique"
[1] "20 ( 28.2 %) : Purpura rhumatoïde"
\end{verbatim}
\end{kframe}
\end{knitrout}


\begin{knitrout}
\definecolor{shadecolor}{rgb}{0.969, 0.969, 0.969}\color{fgcolor}
\includegraphics[width=\maxwidth]{figure/unnamed-chunk-73-1} 

\end{knitrout}

Médiane de survie pour diablète : 5.5 ans [5.25-5.75]

\begin{knitrout}
\definecolor{shadecolor}{rgb}{0.969, 0.969, 0.969}\color{fgcolor}
\includegraphics[width=\maxwidth]{figure/unnamed-chunk-74-1} 

\end{knitrout}

Cox sur age + sexe :
% latex table generated in R 3.3.2 by xtable 1.8-2 package
% Tue Feb 14 12:35:38 2017
\begin{table}[H]
\centering
\begin{tabular}{rrrrrr}
  \hline
 & coef & exp(coef) & se(coef) & z & p \\ 
  \hline
age & 0.08 & 1.08 & 0.01 & 15.76 & 0.00 \\ 
  sex & -0.02 & 0.98 & 0.16 & -0.10 & 0.92 \\ 
   \hline
\end{tabular}
\end{table}


Cox sur age + sexe + dfg :
% latex table generated in R 3.3.2 by xtable 1.8-2 package
% Tue Feb 14 12:35:38 2017
\begin{table}[H]
\centering
\begin{tabular}{rrrrrr}
  \hline
 & coef & exp(coef) & se(coef) & z & p \\ 
  \hline
age & 0.08 & 1.08 & 0.01 & 12.39 & 0.00 \\ 
  sex & -0.00 & 1.00 & 0.20 & -0.02 & 0.98 \\ 
  dfg & 0.02 & 1.02 & 0.03 & 0.72 & 0.47 \\ 
   \hline
\end{tabular}
\end{table}


Cox sur age + sexe + dfg + albini :
% latex table generated in R 3.3.2 by xtable 1.8-2 package
% Tue Feb 14 12:35:38 2017
\begin{table}[H]
\centering
\begin{tabular}{rrrrrr}
  \hline
 & coef & exp(coef) & se(coef) & z & p \\ 
  \hline
age & 0.07 & 1.07 & 0.01 & 9.84 & 0.00 \\ 
  sex & -0.15 & 0.86 & 0.23 & -0.64 & 0.52 \\ 
  dfg & 0.06 & 1.06 & 0.03 & 1.79 & 0.07 \\ 
  ALBINI & -0.05 & 0.95 & 0.01 & -3.57 & 0.00 \\ 
   \hline
\end{tabular}
\end{table}


  %\subsection{Recherche d’un lien entre les évènements infectieux saisonniers et la mise en route de la dialyse}
  
\section{Greffe maladie de Berger}




La table "greffe" concernant la maladie de Berger contient 1738 patients uniques après suppression de ceux qui n'avaient pas de maladie de Berger, de néphropathie à IgA ou de date de greffe. 860 (49.5\%) sont en commun avec la table "globale".
~\\

Dans la table globale avec la table greffe pour le Berger :

% latex table generated in R 3.3.2 by xtable 1.8-2 package
% Tue Feb 14 12:35:38 2017
\begin{table}[H]
\centering
\begin{tabular}{ccc}
  \hline
Date de greffe (global) & Greffe 1 (table greffe) & Total \\ 
  \hline
FALSE & FALSE & 762 \\ 
  FALSE & TRUE & 8 \\ 
  TRUE & FALSE & 98 \\ 
  TRUE & TRUE & 852 \\ 
   \hline
\end{tabular}
\end{table}


% - étant considéré comme greffé (table globale) : table(global_greffe$tx)[2] (round(table(global_greffe$tx)[2]/20455*100,1)\%)
% 
% - ayant une date de greffe (table globale) : table(!is.na(global_greffe$dgrf))[2] (round(table(!is.na(global_greffe$dgrf))[2]/20455*100,1)\%)
% 
% \qquad - dont une greffe de 2010 à 2014 : table(global_greffe$dgrf<as.Date("2015-01-01", origin ="1899-12-30"))[2]
% 
% - ayant eu une greffe sans dialyse (table greffe) : table(global_greffe$METHOn=="Greffe")[2] (round(table(global_greffe$METHOn=="Greffe")[2]/20455*100,1)\%)
% 
% - ayant une date de greffe 1 (table greffe) : table(!is.na(global_greffe$GRF1))[2] (round(table(!is.na(global_greffe$GRF1))[2]/20455*100,1)\%)
% 
% \qquad - dont une greffe de 2010 à 2014 : table(global_greffe$GRF1<as.Date("2015-01-01", origin ="1899-12-30"))[2]
% 
% - ayant une date de greffe 2 (table greffe) : table(!is.na(global_greffe$GRF2))[2] (round(table(!is.na(global_greffe$GRF2))[2]/20455*100,1)\%)
% 
% \qquad - dont une greffe de 2010 à 2014 : table(global_greffe$GRF2<as.Date("2015-01-01", origin ="1899-12-30"))[2]
% 
% - ayant une date de greffe 1 et 2, considéré comme greffé et ayant une date de greffe : table(global_greffe$greffe)[2] (round(table(global_greffe$greffe)[2]/20455*100,1)\%)
% 
% \qquad - dont une greffe de 2010 à 2014 : table(global_greffe$greffe[global_greffe$dgrf<as.Date("2015-01-01", origin ="1899-12-30")]) (round(table(global_greffe$greffe[global_greffe$dgrf<as.Date("2015-01-01", origin ="1899-12-30")])/20455*100,1)\%)

  \subsection{Nombre de greffe par année}
% 
% <<results="asis">>=
% inc_greffe_iga <- matrix(nrow = 4, ncol = 6, dimnames = list(c("Greffe Berger","Pas de greffe Berger","Total France","Cas pour 1M d'hab"),c("2010","2011","2012","2013","2014","2015")))
% inc_greffe_iga[1,] <- c(52, 120, 153, 177, 204, 168)
% inc_greffe_iga[2,] <- c(50482295-52, 50701229-120, 50930784-153, 51192770-177, 51421008-204, 51676545-168)
% inc_greffe_iga[3,] <- c(50482295, 50701229, 50930784, 51192770, 51421008, 51676545)
% inc_greffe_iga[4,] <- round(prop.table(inc_greffe_iga[1:2,],2)*1000000,1)[1,]
% print(xtable(inc_greffe_iga, digits = 1, auto = T), table.placement = "H")
% @
% 
% <<>>=
% chisq.test(inc_greffe_iga[1:2,], correct = F) 
% @

\begin{multicols}{2}
% latex table generated in R 3.3.2 by xtable 1.8-2 package
% Tue Feb 14 12:35:38 2017
\begin{table}[H]
\centering
\begin{tabular}{lrr}
  \hline
Greffe 1 & Fréquence & Pourcentage \\ 
  \hline
2010 & 48 & 5.6 \\ 
  2011 & 114 & 13.3 \\ 
  2012 & 141 & 16.4 \\ 
  2013 & 155 & 18.0 \\ 
  2014 & 176 & 20.5 \\ 
  2015 & 152 & 17.7 \\ 
  2016 & 74 & 8.6 \\ 
  Sum & 860 &  \\ 
   \hline
\end{tabular}
\end{table}
% latex table generated in R 3.3.2 by xtable 1.8-2 package
% Tue Feb 14 12:35:38 2017
\begin{table}[H]
\centering
\begin{tabular}{lrr}
  \hline
Greffe 2 & Fréquence & Pourcentage \\ 
  \hline
2012 &  2 & 15.4 \\ 
  2013 &  2 & 15.4 \\ 
  2014 &  2 & 15.4 \\ 
  2015 &  3 & 23.1 \\ 
  2016 &  4 & 30.8 \\ 
  Sum & 13 &  \\ 
   \hline
\end{tabular}
\end{table}

\end{multicols}

  \subsection{Arrêt de fonction du greffon}

    \subsubsection{Temps entre greffe et arrêt du greffon (sans les patients non en commun avec la table globale)}

% latex table generated in R 3.3.2 by xtable 1.8-2 package
% Tue Feb 14 12:35:38 2017
\begin{table}[H]
\centering
\begin{tabular}{lrrrrrrr}
  \hline
 & Min. & 1st Qu. & Median & Mean & 3rd Qu. & Max. & NA's \\ 
  \hline
Diabète & 0.0 & 0.1 & 5.0 & 13.6 & 23.7 & 62.6 & 11092 \\ 
  GN avec HSF & 0.0 & 0.0 & 0.8 & 8.1 & 11.3 & 40.5 & 1159 \\ 
  GN extra-membraneuse & 0.0 & 1.0 & 3.6 & 14.2 & 24.3 & 51.9 & 425 \\ 
  GN extracapillaire ou endo/extracapillaire & 0.1 & 0.1 & 0.1 & 0.1 & 0.1 & 0.1 & 409 \\ 
  GN membrano-proliférative type 1 & 0.1 & 4.7 & 39.7 & 24.9 & 40.0 & 40.3 & 200 \\ 
  GN membrano-proliférative type 2, dépôts denses & 21.6 & 21.9 & 22.2 & 26.0 & 28.2 & 34.2 & 78 \\ 
  Néphropathie à dépôts d'IgA & 0.0 & 0.0 & 0.8 & 8.2 & 13.8 & 58.7 & 1678 \\ 
  Néphropathie lupique & 0.1 & 3.1 & 8.6 & 13.6 & 24.4 & 33.1 & 201 \\ 
  Polykystose rénale de l adulte & 0.0 & 0.0 & 0.4 & 9.3 & 13.0 & 59.3 & 3068 \\ 
  Purpura rhumatoïde & 15.8 & 16.6 & 17.4 & 17.4 & 18.2 & 19.1 & 69 \\ 
   \hline
\end{tabular}
\end{table}


\begin{knitrout}
\definecolor{shadecolor}{rgb}{0.969, 0.969, 0.969}\color{fgcolor}
\includegraphics[width=\maxwidth]{figure/unnamed-chunk-82-1} 

\includegraphics[width=\maxwidth]{figure/unnamed-chunk-82-2} 

\includegraphics[width=\maxwidth]{figure/unnamed-chunk-82-3} 

\includegraphics[width=\maxwidth]{figure/unnamed-chunk-82-4} 

\includegraphics[width=\maxwidth]{figure/unnamed-chunk-82-5} 

\includegraphics[width=\maxwidth]{figure/unnamed-chunk-82-6} 

\includegraphics[width=\maxwidth]{figure/unnamed-chunk-82-7} 

\includegraphics[width=\maxwidth]{figure/unnamed-chunk-82-8} 

\includegraphics[width=\maxwidth]{figure/unnamed-chunk-82-9} 

\includegraphics[width=\maxwidth]{figure/unnamed-chunk-82-10} 

\end{knitrout}

    \subsubsection{Raison de l'arrêt du greffon}

\subsubsection*{Greffon 1}

% latex table generated in R 3.3.2 by xtable 1.8-2 package
% Tue Feb 14 12:35:42 2017
\begin{table}[H]
\centering
\begin{tabular}{lrr}
  \hline
Diabète & Frequence & Pourcentage \\ 
  \hline
107 & 24 & 29.3 \\ 
  104 & 21 & 25.6 \\ 
  103 & 8 & 9.8 \\ 
  352 & 6 & 7.3 \\ 
  117 & 3 & 3.7 \\ 
  216 & 3 & 3.7 \\ 
  101 & 2 & 2.4 \\ 
  105 & 2 & 2.4 \\ 
  705 & 2 & 2.4 \\ 
  000 & 1 & 1.2 \\ 
  102 & 1 & 1.2 \\ 
  106 & 1 & 1.2 \\ 
  150 & 1 & 1.2 \\ 
  201 & 1 & 1.2 \\ 
  309 & 1 & 1.2 \\ 
  351 & 1 & 1.2 \\ 
  402 & 1 & 1.2 \\ 
  701 & 1 & 1.2 \\ 
  707 & 1 & 1.2 \\ 
  911 & 1 & 1.2 \\ 
   \hline
\end{tabular}
\end{table}
% latex table generated in R 3.3.2 by xtable 1.8-2 package
% Tue Feb 14 12:35:42 2017
\begin{table}[H]
\centering
\begin{tabular}{lrr}
  \hline
GN avec HSF & Frequence & Pourcentage \\ 
  \hline
107 & 10 & 41.7 \\ 
  104 & 4 & 16.7 \\ 
  106 & 3 & 12.5 \\ 
  103 & 2 & 8.3 \\ 
  101 & 1 & 4.2 \\ 
  105 & 1 & 4.2 \\ 
  111 & 1 & 4.2 \\ 
  201 & 1 & 4.2 \\ 
  706 & 1 & 4.2 \\ 
   \hline
\end{tabular}
\end{table}
% latex table generated in R 3.3.2 by xtable 1.8-2 package
% Tue Feb 14 12:35:42 2017
\begin{table}[H]
\centering
\begin{tabular}{lrr}
  \hline
GN extra-membraneuse & Frequence & Pourcentage \\ 
  \hline
104 & 5 & 41.7 \\ 
  107 & 4 & 33.3 \\ 
  103 & 2 & 16.7 \\ 
  707 & 1 & 8.3 \\ 
   \hline
\end{tabular}
\end{table}
% latex table generated in R 3.3.2 by xtable 1.8-2 package
% Tue Feb 14 12:35:42 2017
\begin{table}[H]
\centering
\begin{tabular}{lrr}
  \hline
GN extracapillaire ou endo/extracapillaire & Frequence & Pourcentage \\ 
  \hline
107 & 2 & 100 \\ 
   \hline
\end{tabular}
\end{table}
% latex table generated in R 3.3.2 by xtable 1.8-2 package
% Tue Feb 14 12:35:42 2017
\begin{table}[H]
\centering
\begin{tabular}{lrr}
  \hline
GN membrano-proliférative type 1 & Frequence & Pourcentage \\ 
  \hline
104 & 2 & 40 \\ 
  107 & 2 & 40 \\ 
  101 & 1 & 20 \\ 
   \hline
\end{tabular}
\end{table}
% latex table generated in R 3.3.2 by xtable 1.8-2 package
% Tue Feb 14 12:35:42 2017
\begin{table}[H]
\centering
\begin{tabular}{lrr}
  \hline
GN membrano-proliférative type 2, dépôts denses & Frequence & Pourcentage \\ 
  \hline
104 & 1 & 33.3 \\ 
  111 & 1 & 33.3 \\ 
  990 & 1 & 33.3 \\ 
   \hline
\end{tabular}
\end{table}
% latex table generated in R 3.3.2 by xtable 1.8-2 package
% Tue Feb 14 12:35:42 2017
\begin{table}[H]
\centering
\begin{tabular}{lrr}
  \hline
Néphropathie à dépôts d'IgA & Frequence & Pourcentage \\ 
  \hline
107 & 23 & 54.8 \\ 
  104 & 7 & 16.7 \\ 
  103 & 5 & 11.9 \\ 
  117 & 3 & 7.1 \\ 
  106 & 2 & 4.8 \\ 
  712 & 2 & 4.8 \\ 
   \hline
\end{tabular}
\end{table}
% latex table generated in R 3.3.2 by xtable 1.8-2 package
% Tue Feb 14 12:35:42 2017
\begin{table}[H]
\centering
\begin{tabular}{lrr}
  \hline
Néphropathie lupique & Frequence & Pourcentage \\ 
  \hline
103 & 2 & 33.3 \\ 
  107 & 2 & 33.3 \\ 
  104 & 1 & 16.7 \\ 
  910 & 1 & 16.7 \\ 
   \hline
\end{tabular}
\end{table}
% latex table generated in R 3.3.2 by xtable 1.8-2 package
% Tue Feb 14 12:35:42 2017
\begin{table}[H]
\centering
\begin{tabular}{lrr}
  \hline
Polykystose rénale de l adulte & Frequence & Pourcentage \\ 
  \hline
107 & 33 & 43.4 \\ 
  104 & 13 & 17.1 \\ 
  101 & 11 & 14.5 \\ 
  117 & 5 & 6.6 \\ 
  103 & 3 & 3.9 \\ 
  911 & 3 & 3.9 \\ 
  105 & 1 & 1.3 \\ 
  111 & 1 & 1.3 \\ 
  215 & 1 & 1.3 \\ 
  256 & 1 & 1.3 \\ 
  352 & 1 & 1.3 \\ 
  712 & 1 & 1.3 \\ 
  909 & 1 & 1.3 \\ 
  990 & 1 & 1.3 \\ 
   \hline
\end{tabular}
\end{table}
% latex table generated in R 3.3.2 by xtable 1.8-2 package
% Tue Feb 14 12:35:42 2017
\begin{table}[H]
\centering
\begin{tabular}{lrr}
  \hline
Purpura rhumatoïde & Frequence & Pourcentage \\ 
  \hline
103 & 1 & 50 \\ 
  402 & 1 & 50 \\ 
   \hline
\end{tabular}
\end{table}


\subsubsection*{Greffon 2}

% latex table generated in R 3.3.2 by xtable 1.8-2 package
% Tue Feb 14 12:35:42 2017
\begin{table}[H]
\centering
\begin{tabular}{rrrr}
  \hline
 & Défaillance primaire & Rejet aigu & Compli vasc greffon \\ 
  \hline
Diabète &   0 &   0 &   1 \\ 
  GN avec HSF &   0 &   0 &   0 \\ 
  GN extra-membraneuse &   0 &   1 &   0 \\ 
  GN extracapillaire ou endo/extracapillaire &   0 &   0 &   0 \\ 
  GN membrano-proliférative type 1 &   0 &   0 &   1 \\ 
  GN membrano-proliférative type 2, dépôts denses &   0 &   0 &   0 \\ 
  Néphropathie à dépôts d'IgA &   0 &   0 &   2 \\ 
  Néphropathie lupique &   0 &   0 &   0 \\ 
  Polykystose rénale de l adulte &   1 &   0 &   1 \\ 
  Purpura rhumatoïde &   0 &   0 &   0 \\ 
   \hline
\end{tabular}
\end{table}


    \subsubsection{Pourcentage d'arrêt de greffon/greffe/an}

Maladie de Berger : 42 (4.9\%)

% latex table generated in R 3.3.2 by xtable 1.8-2 package
% Tue Feb 14 12:35:42 2017
\begin{table}[H]
\centering
\begin{tabular}{lrrrrr}
  \hline
Arrêt greffon & 2010 & 2011 & 2012 & 2013 & 2014 \\ 
  \hline
FALSE & 94 & 93.8 & 95.3 & 96.1 & 97.2 \\ 
  TRUE & 6 & 6.2 & 4.7 & 3.9 & 2.8 \\ 
   \hline
\end{tabular}
\end{table}


Toutes les pathologies : 255 (6\%)

% latex table generated in R 3.3.2 by xtable 1.8-2 package
% Tue Feb 14 12:35:42 2017
\begin{table}[H]
\centering
\begin{tabular}{lrrrrr}
  \hline
Arrêt greffon & 2010 & 2011 & 2012 & 2013 & 2014 \\ 
  \hline
FALSE & 91.4 & 93.7 & 94.6 & 95.4 & 95.5 \\ 
  TRUE & 8.6 & 6.3 & 5.4 & 4.6 & 4.5 \\ 
   \hline
\end{tabular}
\end{table}


\textcolor{red}{Mais de moins en moins de recul}

    \subsubsection{Temps d'arrêt du greffon selon la raison d'arrêt (en mois)}

Maladie de Berger

% latex table generated in R 3.3.2 by xtable 1.8-2 package
% Tue Feb 14 12:35:42 2017
\begin{table}[H]
\centering
\begin{tabular}{lrrrrrr}
  \hline
 & Min. & 1st Qu. & Median & Mean & 3rd Qu. & Max. \\ 
  \hline
103 & 6.2 & 13.9 & 14.8 & 22.0 & 16.3 & 58.7 \\ 
  104 & 1.5 & 7.1 & 20.1 & 19.5 & 25.8 & 49.3 \\ 
  106 & 13.4 & 15.4 & 17.5 & 17.5 & 19.5 & 21.6 \\ 
  107 & 0.0 & 0.0 & 0.1 & 1.3 & 0.2 & 20.1 \\ 
  117 & 0.9 & 1.0 & 1.0 & 1.3 & 1.4 & 1.8 \\ 
  712 & 3.8 & 8.7 & 13.7 & 13.7 & 18.6 & 23.6 \\ 
   \hline
\end{tabular}
\end{table}



% Pourcentage d'arrêt de greffon/greffe/an pour la table greffe : table(!is.na(greffe$GRF1), !is.na(greffe$ARF1))[2] (round(prop.table(table(!is.na(greffe$GRF1), !is.na(greffe$ARF1)))*100,1)[2]\%)
% 
% <<>>=
% for (i in 2010:2014) {
%   print(i)
%   print(round(prop.table(table(!is.na(greffe$GRF1[global_greffe$anirt==i]), !is.na(greffe$ARF1[global_greffe$anirt==i]))[2,])*100,1))
% }
% @

  \subsection{Inscription liste de greffe}

Deux variables : "date d'inscription sur liste d'attente greffe" et "Inscription sur liste d'attente (selon REIN)" avec incohérences (certains patients avec une date et considérés comme non inscrit).

Nombre de patients ayant une date d'inscription (Berger) : 1235 (71.8\%)

\begin{multicols}{2}
% latex table generated in R 3.3.2 by xtable 1.8-2 package
% Tue Feb 14 12:35:42 2017
\begin{table}[H]
\centering
\begin{tabular}{lrr}
  \hline
 & Non greffé & Greffé \\ 
  \hline
Non inscrit & 100.0 & 0.0 \\ 
  Inscrit & 45.7 & 54.3 \\ 
   \hline
\end{tabular}
\caption{Diabète} 
\end{table}
% latex table generated in R 3.3.2 by xtable 1.8-2 package
% Tue Feb 14 12:35:42 2017
\begin{table}[H]
\centering
\begin{tabular}{lrr}
  \hline
 & Non greffé & Greffé \\ 
  \hline
Non inscrit & 100.0 & 0.0 \\ 
  Inscrit & 31.7 & 68.3 \\ 
   \hline
\end{tabular}
\caption{GN avec HSF} 
\end{table}
% latex table generated in R 3.3.2 by xtable 1.8-2 package
% Tue Feb 14 12:35:42 2017
\begin{table}[H]
\centering
\begin{tabular}{lrr}
  \hline
 & Non greffé & Greffé \\ 
  \hline
Non inscrit & 100.0 & 0.0 \\ 
  Inscrit & 30.4 & 69.6 \\ 
   \hline
\end{tabular}
\caption{GN extra-membraneuse} 
\end{table}
% latex table generated in R 3.3.2 by xtable 1.8-2 package
% Tue Feb 14 12:35:42 2017
\begin{table}[H]
\centering
\begin{tabular}{lrr}
  \hline
 & Non greffé & Greffé \\ 
  \hline
Non inscrit & 100.0 & 0.0 \\ 
  Inscrit & 45.9 & 54.1 \\ 
   \hline
\end{tabular}
\caption{GN extracapillaire ou endo/extracapillaire} 
\end{table}
% latex table generated in R 3.3.2 by xtable 1.8-2 package
% Tue Feb 14 12:35:42 2017
\begin{table}[H]
\centering
\begin{tabular}{lrr}
  \hline
 & Non greffé & Greffé \\ 
  \hline
Non inscrit & 100.0 & 0.0 \\ 
  Inscrit & 24.0 & 76.0 \\ 
   \hline
\end{tabular}
\caption{GN membrano-proliférative type 1} 
\end{table}
% latex table generated in R 3.3.2 by xtable 1.8-2 package
% Tue Feb 14 12:35:42 2017
\begin{table}[H]
\centering
\begin{tabular}{lrr}
  \hline
 & Non greffé & Greffé \\ 
  \hline
Non inscrit & 100.0 & 0.0 \\ 
  Inscrit & 25.5 & 74.5 \\ 
   \hline
\end{tabular}
\caption{GN membrano-proliférative type 2, dépôts denses} 
\end{table}
% latex table generated in R 3.3.2 by xtable 1.8-2 package
% Tue Feb 14 12:35:42 2017
\begin{table}[H]
\centering
\begin{tabular}{lrr}
  \hline
 & Non greffé & Greffé \\ 
  \hline
Non inscrit & 99.8 & 0.2 \\ 
  Inscrit & 22.5 & 77.5 \\ 
   \hline
\end{tabular}
\caption{Néphropathie à dépôts d'IgA} 
\end{table}
% latex table generated in R 3.3.2 by xtable 1.8-2 package
% Tue Feb 14 12:35:42 2017
\begin{table}[H]
\centering
\begin{tabular}{lrr}
  \hline
 & Non greffé & Greffé \\ 
  \hline
Non inscrit & 100.0 & 0.0 \\ 
  Inscrit & 32.2 & 67.8 \\ 
   \hline
\end{tabular}
\caption{Néphropathie lupique} 
\end{table}
% latex table generated in R 3.3.2 by xtable 1.8-2 package
% Tue Feb 14 12:35:42 2017
\begin{table}[H]
\centering
\begin{tabular}{lrr}
  \hline
 & Non greffé & Greffé \\ 
  \hline
Non inscrit & 99.9 & 0.1 \\ 
  Inscrit & 22.0 & 78.0 \\ 
   \hline
\end{tabular}
\caption{Polykystose rénale de l adulte} 
\end{table}
% latex table generated in R 3.3.2 by xtable 1.8-2 package
% Tue Feb 14 12:35:42 2017
\begin{table}[H]
\centering
\begin{tabular}{lrr}
  \hline
 & Non greffé & Greffé \\ 
  \hline
Non inscrit & 100.0 & 0.0 \\ 
  Inscrit & 14.3 & 85.7 \\ 
   \hline
\end{tabular}
\caption{Purpura rhumatoïde} 
\end{table}

\end{multicols}

  \subsection{Décès au cours de la greffe}

En prenant en compte les patients ayant une date de greffe :

- Berger : 13/860 (1.5\%) patients sont décédés au cours de la greffe 1.

- Diabète : 87/1096 (7.9\%) patients sont décédés au cours de la greffe 1.

- PKRD : 35/1609 (2.2\%) patients sont décédés au cours de la greffe 1.

- GNC : choisir els pathologies intégrées
%length(global_greffe$DECES1[!is.na(global_greffe$DECES1) & global_greffe$gnc==1])/length(global_greffe$DECES1[global_greffe$gnc==1 & !is.na(global_greffe$GRF1)]) (round(length(global_greffe$DECES1[!is.na(global_greffe$DECES1) & global_greffe$gnc==1])/length(global_greffe$DECES1[global_greffe$gnc==1 & !is.na(global_greffe$GRF1)])*100,1)\%) patients sont décédés au cours de la greffe 1.

\begin{knitrout}
\definecolor{shadecolor}{rgb}{0.969, 0.969, 0.969}\color{fgcolor}
\includegraphics[width=\maxwidth]{figure/unnamed-chunk-89-1} 

\end{knitrout}


\begin{knitrout}
\definecolor{shadecolor}{rgb}{0.969, 0.969, 0.969}\color{fgcolor}
\includegraphics[width=\maxwidth]{figure/unnamed-chunk-90-1} 

\end{knitrout}


\begin{knitrout}
\definecolor{shadecolor}{rgb}{0.969, 0.969, 0.969}\color{fgcolor}
\includegraphics[width=\maxwidth]{figure/unnamed-chunk-91-1} 

\end{knitrout}
















\end{document}
